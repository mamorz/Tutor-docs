\section{Abstrakte Datenstrukturen}

\begin{frame}{Abstrakte Datenstrukturen}
    \setbeamercovered{transparent}
    \begin{block}{Definition}
        Ein Abstrakter Datentyp (ADT) ist ein Verbund von Daten zusammen 
        mit der Definition aller zulässigen Operationen, die auf sie zugreifen. 
    \end{block}

    \pause

    \begin{block}{Genauer}
        \begin{itemize}
            \item Definiert nur Schnittstellen auf diese Daten
            \item Kapselt dadurch die Daten nach außen
            \item Funktioniert im inneren komplexer und implementierungsabhängig
        \end{itemize}
    \end{block}

    \pause

    \begin{block}{Beispiele}
        \begin{itemize}
            \item Listen (verschiedene Arten)
            \item Graphen (Und alle „Unterdatenstrukturen“)
            \item Viel mehr…
        \end{itemize}
    \end{block}
\end{frame}