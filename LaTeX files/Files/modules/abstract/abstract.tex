\section{Abstrakt}

\begin{frame}{Abstrakte Klassen \& Methoden}
    \begin{block}{Allgemein}
        Mit dem Schlüsselwort \texttt{abstract} kann man eine Klasse als „reine Oberklasse“ deklarieren

        \begin{itemize}
            \item Abstrakte Klassen können nicht direkt instanziiert werden
            \item Keine oder unvollständige Implementierung
            \item Methoden können ebenfalls als \texttt{abstract} deklariert werden, wenn sie in der abstrakten Klasse (noch) nicht implementiert werden
            \item Eine Klasse mit abstrakten Methoden muss abstrakt sein
            \item Unterschied zu Interfaces?
        \end{itemize}
    \end{block}
\end{frame}

\begin{frame}{Abstrakte Klassen \& Methoden}
    \begin{exampleblock}{Beispiel}
        \lstinputlisting[basicstyle=\small]{../modules/abstract/src/abstract.java}
    \end{exampleblock}

    \begin{itemize}
        \item Die Frucht kann nicht direkt instanziiert werden
        \item \texttt{consume} wird nicht implementiert, nur die Signatur
    \end{itemize}
\end{frame}
