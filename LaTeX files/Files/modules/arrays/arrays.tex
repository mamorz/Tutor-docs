\graphicspath{{../modules/arrays/img/}}

\section{Arrays}

\begin{frame}{Arrays - Definition}
    \begin{block}{Definition}

        \pause

        \begin{itemize}
            \item geordnete (nicht sortierte!) Sammlung von Elementen \dots
            \item \dots eines Typs: Alle Elemente haben den gleichen Typ
            \item Länge: Anzahl der Elemente ($n$)
            \item Nummerierung: Von $0$ bis $n - 1$ (Indizierung started bei \textbf{0})
        \end{itemize}
    \end{block}

    \pause

    \begin{exampleblock}{Beispiel}
        \begin{center}
            \includegraphics[scale=0.4]{arraysample}
        \end{center}
    \end{exampleblock}
\end{frame}

\begin{frame}{Arrays - Deklaration}
    \begin{block}{Deklaration}
        Es gibt zwei Varianten, Arrays zu deklarieren \dots

        \pause

        \begin{itemize}
            \item \texttt{Typ1[] arrayName = new Typ1[anzahlElemente];} \\
                  \hspace{1em}\texttt{z.B. float[] a = new float[8];}

                  \pause

            \item \texttt{Typ2[] arrayName = \{element1, element2, \dots\};} \\
                  \hspace{1em}\texttt{z.B. int[] primeNumbers = \{2, 3, 5, 7, 11\};}
        \end{itemize}
    \end{block}

    \pause

    \begin{alertblock}{Häufiger Fehler}
        \lstinputlisting[basicstyle=\small]{../modules/arrays/src/frequent-error.java}
        Wieso?

        \pause

        \textbf{Keine Zahl im Datentyp!}
    \end{alertblock}
\end{frame}

\begin{frame}{Arrays - Selektion}
    \begin{block}{Elemente selektieren}
        \begin{itemize}
            \item \texttt{arrayName[elementIndex];}
            \item \texttt{primeNumbers[0]} liefert das erste Element

                  \pause

            \item \textit{Was liefern:} \texttt{primeNumbers[1]} \textit{und}
                  \texttt{primeNumbers[4]}?

                  \pause

            \item \textit{Was ist mit } \texttt{primeNumbers[5]}?
        \end{itemize}
    \end{block}
\end{frame}

\begin{frame}{Arrays - Werte setzen}
    \begin{block}{Wertzuweisung}
        \begin{itemize}
            \item \texttt{arrayName[elementIndex] = wert;}
            \item \texttt{ z.B. primeNumbers[0] = 13;}
                  \\ \ \\
                  Wie sieht jetzt die Belegung aus?
        \end{itemize}
    \end{block}

    \pause

    \begin{block}{Arraylänge}
        Die Länge eines Array erhält man mit
        \begin{itemize}
            \item \texttt{arrayName.length; // Achtung, Attribut, keine Methode!} \\
                  \hspace{1em}\texttt{z.B. primeNumbers.length; // liefert den Wert 5}
        \end{itemize}
    \end{block}
\end{frame}

\begin{frame}{Arrays - iterieren}
    \begin{exampleblock}{Iterieren über Arrays}
        \lstinputlisting[basicstyle=\small]{../modules/arrays/src/iterate-array.java}
    \end{exampleblock}
\end{frame}

\begin{frame}{Arrays - Objektlike}
    \begin{alertblock}{Achtung!}
        Arrays werden intern wie Objekte behandelt! \\
        Also bitte beachten, wann Ihr eine Referenz und wann einen Wert erhaltet!
    \end{alertblock}
\end{frame}

\begin{frame}{Aufgabe - Arrays}
    \begin{block}{Arrays - einfach Array}
        Schreiben Sie eine Methode \dots
        \begin{itemize}
            \item \textbf{public static int arraySum(int[] array)}, \\
                  die die Summe der Zahlen des übergebenen Arrays als Rückgabewert hat.
            \item \textbf{public static double average(int[] array)}, \\
                  die den durchschnittlichen Wert der Zahlen des übergebenen Arrays als Rückgabewert hat.
        \end{itemize}
    \end{block}
\end{frame}
