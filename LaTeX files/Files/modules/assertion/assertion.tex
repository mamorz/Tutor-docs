\section{Assertions}

\begin{frame}{Assertions}
    \begin{block}{Allgemein}
        \textbf{Zusicherungen} (engl. Assertions) sind ein weiteres wichtiges Hilfsmittel für die Entwicklung von Software. \\
        Sie werden verwendet, um

        \begin{itemize}
            \item Restriktionen für Parameter und globale Variablen von Methoden anzugeben (Vorbedingung)
            \item den Effekt von Methoden formal zu beschreiben (Nachbedingung)
            \item an zentralen Programmpunkten wichtige Eigenschaften (z.B. mögliche Werte von Variablen) explizit festzuhalten
        \end{itemize}

        Mit Hilfe von Zusicherungen ist es möglich Korrektheitsaussagen bzgl. eines gegeben Programmes mathematisch zu beweisen.
    \end{block}

    \begin{center}
        \textbf{Keine Verwendung in diesem Modul!}
    \end{center}
\end{frame}

\begin{frame}{Assertions}
    \begin{block}{Java}
        \begin{itemize}
            \item Java bietet die Möglichkeit Assertions direkt in den Programmtext zu schreiben (\texttt{assert})
            \item Automatisierte Überprüfung, aber nur während der Laufzeit
            \item Eingeschränkt durch Java-Syntax im Vergleich zu Kommentaren
            \item Assertions werden standardmäßig nicht ausgewertet. Sie müssen beim Programmstart mit \texttt{java -ea MyClass} aktiviert werden (\textbf{e}nable \textbf{a}ssertions)
        \end{itemize}
    \end{block}

    \begin{block}{Syntax}
        \texttt{assert Boolesche Bedingung;} \\
        \texttt{assert Boolesche Bedingung: "Detaillierte Fehlermeldung";}
    \end{block}
\end{frame}

\begin{frame}{Assertions}
    \begin{exampleblock}{Beispiel}
        \lstinputlisting[basicstyle=\small]{../modules/assertion/src/multiply-scalar.java}
    \end{exampleblock}
\end{frame}

\begin{frame}{Assertions}
    \begin{block}{Eigenschaften}
        Eine Zusicherung gibt eine Eigenschaft an, die bei der Ausführung des Programms an der entsprechenden Stelle erfüllt sein muss.

        \begin{itemize}
            \item Eine \textbf{Precondition} muss vor der Abarbeitung des entsprechenden Methodenrumpfs erfüllt sein
            \item Eine \textbf{Postcondition} muss erfüllt sein, nachdem die Methode abgearbeitet wurde
            \item Eine \textbf{Instanz-Invariante} ist eine Zusicherung, die sowohl vor als auch nach jedem Methodenaufruf (Ausnahme: private Hilfsmethoden) des zugehörigen Objekts gültig sein muss. \\
                  Beispiel: \texttt{int length; assert length >= 0;}
            \item Eine \textbf{Schleifen-Invariante} ist eine Zusicherung, die am Anfang und Ende eines jeden Durchlaufes der zugehörigen Schleife erfüllt sein muss
        \end{itemize}
    \end{block}
\end{frame}

\begin{frame}{Assertions}
    \begin{exampleblock}{assert vs if}
        \lstinputlisting[basicstyle=\small]{../modules/assertion/src/assert-if.java}
    \end{exampleblock}
\end{frame}

\begin{frame}{Assertions}
    \begin{block}{assert}
        \begin{itemize}
            \item Dokumentation: Nur Überprüfungszweck
            \item Zur eigentlichen Laufzeit abschaltbar
        \end{itemize}
    \end{block}

    \begin{block}{if}
        \begin{itemize}
            \item wird immer ausgeführt
            \item ggf. teuer zur Laufzeit
        \end{itemize}
    \end{block}
\end{frame}
