\graphicspath{{../modules/command-system/img/}}
\section{Commands}

\begin{frame}{Ein kleines Musterlösungssystem}
    \begin{block}{Siehe}
        \texttt{SimpleCommandSystem.java}\\
        $\rightarrow$ Einfache Lösung\\
        $\rightarrow$ Wird sehr schnell unübersichtlich bei vielen Commands
    \end{block}
\end{frame}

\begin{frame}{Das ''offizielle'' Command System}
    \setbeamercovered{invisible}
    \begin{block}{Was ist das?}
        \begin{itemize}
            \item Eines der \textit{twenty-three well-known gang of four design patterns}
            \item Mehr zu diesen Entwurfsmustern in SWT I
        \end{itemize}
    \end{block}

    \pause

    \begin{block}{Vorteile}
        \begin{itemize}
            \item Sorgt für gute Struktur und Schafft Übersichtlichkeit
            \item Man kann Befehlsausführungen verwalten und so sehr leicht eine undo Funktion implementieren
            \item Separiert Behfehlsstruktur vom Rest der Programmlogik
            \aitem Neue commands können leicht ergänzt werden
            \aitem Lose Kopplung von Befehlen zum Rest des Programmes
        \end{itemize}
    \end{block}
\end{frame}

\begin{frame}{Das ''offizielle'' Command System}
    \begin{block}{Bestandteile}
        \begin{itemize}
            \item Ein \textit{Command} Interface
            \begin{itemize}
                \item Hat eine \lstinline{execute()} Methode
                \item Hat idr. Eine Methode zum Namen erhalten
            \end{itemize}
            \item Eine konkrete Klasse pro Command
            \begin{itemize}
                \item Implementiert das Command Interface
                \item Implementiert alle Methoden aus dem Interface auf passende Weise
                \item Hält Referenz auf die nötige(n) Receiver Klasse(n) um Methoden auf diesen Aufrufen zu können
            \end{itemize}
            \item Eine \textit{Caller} Klasse
            \begin{itemize}
                \item Ruft den passenden Command auf
                \item Kennt die konkreten Command Klassen nicht, nur das Interface
                \item Hält idr. eine Listee oder ähnliches aller möglichen Commands
            \end{itemize}
            \item Eine oder mehrere \textit{Receiver} Klasse(n)
            \begin{itemize}
                \item Die Klasse an der tatsächlich etwas geändert wird
                \item Bei einem Spiel könnte das zum Beispiel die Spiel Klasse sein um die nächste Runde einzuleiten
            \end{itemize}
        \end{itemize}
    \end{block}
\end{frame}

\begin{frame}
    \begin{center}
        \includegraphics[scale=0.7]{command_system_uml.jpg}    
    \end{center}
    \footnotesize By Vanderjoe - Own work, CC BY-SA 4.0, https://commons.wikimedia.org/w/index.php?curid=62530466
\end{frame}

