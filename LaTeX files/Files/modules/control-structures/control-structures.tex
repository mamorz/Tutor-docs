\graphicspath{{../modules/control-structures/img/}}

\section{Kontrollstrukturen}

\subsection{Verzweigungen}

\begin{frame}[fragile]{Verzweigungen}
    \begin{block}{if}
        \begin{itemize}
            \item Häufig soll ein Teil des Programmes nicht immer, sondern nur manchmal ausgeführt werden
            \item \texttt{if} erlaubt es, die Ausführung basierend auf Bedingungen zu steuern
        \end{itemize}
    \end{block}

    \pause

    \begin{exampleblock}{Ein Spiel für Ältere}
        \begin{lstlisting}
int age = userAge;
if(age < 16) {
    System.out.println("Too young :(");
} else {
    System.out.println("Alright!");
}
        \end{lstlisting}
    \end{exampleblock}
\end{frame}

\begin{frame}[fragile]{Verzweigungen}
    \begin{block}{If-Kontrollflussgraph}
        \scalebox{0.8}{%
            \input{../modules/control-structures/src/if.latex}
        }
    \end{block}
\end{frame}

\begin{frame}[fragile]{Verzweigungen}
    \begin{exampleblock}{Ein Spiel für Ältere}
        \scalebox{0.8}{%
            \input{../modules/control-structures/src/if-game.latex}
        }
    \end{exampleblock}
\end{frame}

\begin{frame}[fragile]{Verzweigungen}
    \begin{block}{\texttt{else if}}
        \begin{itemize}
            \item Neben \texttt{if} und \texttt{else} gibt es noch \texttt{else if}
            \item Kann wie ein geschachteltes \texttt{else} -> \texttt{if} angesehen werden
            \item Nur \textbf{ein} Block des ifs wird betreten
        \end{itemize}
    \end{block}

    \pause

    \begin{exampleblock}{\texttt{else if}}
        \begin{lstlisting}[showstringspaces=false]
int age = userAge;
if(age < 13) {
    System.out.println("Have a look at this 'outside'!");
} else if(age < 16) {
    System.out.println("Too young :(");
} else {
    System.out.println("Alright!");
}
        \end{lstlisting}
    \end{exampleblock}
\end{frame}

\begin{frame}{Aufgaben zu if}
    \begin{block}{Altersabfrage}
        \small
        Für eine Spielefirma soll ein totsicheres System zum Erkennen minderjähriger Nutzer entwickelt werden.
        Sie haben bereits ein Fenster implementiert, dass den Nutzer nach dem Alter fragt und jede Eingabe zwischen 0 und 13 ablehnt.
        \\ \ \\
        Nun ist den Managern jedoch zu Ohren gekommen,
        dass viele Spieler einfach \emph{irgendeine} riesige Zahl eintragen, was sie nun verhindern müssen.
        \\ \ \\
        Ihr Programm bekommt als Eingabe einen \texttt{int alter} und soll
        \texttt{System.out.println} für Ausgaben nutzen.

        \begin{itemize}
            \item Schreiben Sie ein Programm, dass jedes Alter größer als 127 ablehnt.
            \item Erweitern Sie das Programm, sodass für das Alter 1337 „Nicht schlecht“ ausgegeben wird.
            \item Passen Sie ihre erste Bedingung so an, dass auch Alter < 0 abgelehnt werden.
        \end{itemize}
    \end{block}
\end{frame}

\begin{frame}{Switch}
    \begin{block}{Motivation}
        \begin{itemize}
            \item Lange \texttt{if} - \texttt{else if} Ketten werden schnell unübersichtlich
            \item Wenn mehrere Werte erlaubt sind, werden die Bedingungen schnell groß
            \item Lösung? Switch!
        \end{itemize}
    \end{block}
\end{frame}

\begin{frame}[fragile]{Switch}
    \begin{block}{Aufbau}
        \begin{lstlisting}
int x = 5;
switch(x) {
case 1: // "falle durch" zum naechsten
case 2: // "falle durch" zum naechsten
// ...
case 5:
    // Behandle 5 und alle, die "durchgefallen" sind (1,2,...)
    break; // verlasse den Fall. Stoppe Durchfallen
case 6:
    // Behandle NUR die 6, da nichts darueber durchfaellt
    break;
default:
    // Wird aufgerufen, falls kein anderer Fall zutrifft
    break;
}
        \end{lstlisting}
    \end{block}
\end{frame}

\begin{frame}[fragile]{Switch}
    \begin{exampleblock}{Beispiel}
        \vspace*{-0.5em}
        \lstinputlisting{../modules/control-structures/src/switch.java}
    \end{exampleblock}
\end{frame}

\begin{frame}{Switch}
    \begin{block}{Beschränkungen}
        \begin{itemize}
            \item Funktioniert nur mit Ganzzahlen, \texttt{String} und Enums
        \end{itemize}
    \end{block}

    \begin{alertblock}{Fallthrough beachten}
        \begin{itemize}
            \item Das Durchfallen in darunterliegende Fälle ist oft ein Programmierfehler \\
                  $\Rightarrow$ Aufpassen und \texttt{break} nicht vergessen!
        \end{itemize}
    \end{alertblock}
\end{frame}

\setbeamercovered{transparent}

\begin{frame}[fragile]{Conditional Operator \texttt{? :}}
    \begin{block}{Motivation}
        \begin{itemize}
            \item \begin{lstlisting}
int x;
if(bedingung) {
    x = 200;
} else {
    x = 500;
}
            \end{lstlisting}
            \item ist relativ lang und verschleiert damit die Logik
        \end{itemize}
    \end{block}

    \pause

    \begin{block}{Allgemein}
        \begin{itemize}
            \item Auch bekannt als „ternary Operator“
            \item Aufbau:\ \ \texttt{Bedingung ? <falls wahr> : <falls falsch>}\\
                  Dabei ist die Bedingung ein \texttt{boolean} und \texttt{<falls wahr/falsch>} sind Ausdrücke.
        \end{itemize}
    \end{block}
\end{frame}

\begin{frame}[fragile]{Conditional Operator \texttt{? :}}
    \begin{exampleblock}{Vorheriges Beispiel}
        \begin{lstlisting}
int x;
if(bedingung) {
    x = 200;
} else {
    x = 500;
}
        \end{lstlisting}
    \end{exampleblock}

    \begin{exampleblock}{Mit dem Conditional Operator}
        \begin{lstlisting}
int x = z ? 200 : 500;
        \end{lstlisting}
    \end{exampleblock}
\end{frame}

\begin{frame}[fragile]{Conditional Operator \texttt{? :}}
    \begin{block}{Wann benutzt man ihn?}
        \begin{itemize}
            \item Wenn die Bedingung kurz ist
            \item Wenn man in dem if nur eine Variable setzt
            \item Als Rückgabe-Ausdruck in Methoden
        \end{itemize}
    \end{block}

    \begin{alertblock}{Wann benutzt man ihn nicht?}
        \begin{itemize}
            \item Wenn die Bedingung zu lang ist
            \item Wenn man ihn verschachtelt: \\
                  \begin{lstlisting}
int x = z ? y ? 20 : 10 : 5;
                 \end{lstlisting}
            \item \textbf{Sehr vorsichtig mit diesem Operator sein!}
            \item Checkstyle sagt wahrscheinlich nein
        \end{itemize}
    \end{alertblock}
\end{frame}

\setbeamercovered{invisible}

\subsection{Schleifen}

\begin{frame}{Schleifen}
    \begin{block}{Wozu?}
        Mit Schleifen kann Programmcode wiederholt ausgeführt werden. \\
        Java kennt folgende vier Schleifentypen:

        \begin{itemize}
            \item while
            \item do \dots while
            \item for
            \item for-each
        \end{itemize}

        Grundsätzlich kann jeder Schleifentyp in jeden anderen überführt werden.
    \end{block}
\end{frame}

\begin{frame}{while}
    \begin{block}{while}
        \lstinputlisting[basicstyle=\small]{../modules/control-structures/src/while.java}
    \end{block}

    \begin{block}{Hinweise}
        \begin{itemize}
            \item Initialisierung notwendiger Variablen ist Aufgabe des Entwicklers
            \item Bedingung wird \textbf{vor} jedem Schleifendurchlauf geprüft
        \end{itemize}
    \end{block}
\end{frame}

\begin{frame}{while - Beispiel}
    \begin{exampleblock}{Beispiel}
        \lstinputlisting[basicstyle=\small]{../modules/control-structures/src/while-example.java}
    \end{exampleblock}
\end{frame}

\begin{frame}{do \dots while}
    \begin{block}{do \dots while}
        \lstinputlisting[basicstyle=\small]{../modules/control-structures/src/do-while.java}
    \end{block}

    \begin{block}{Hinweise}
        \begin{itemize}
            \item Bedingung wird \textbf{nach} jedem Schleifendurchlauf geprüft
            \item Schleife wird mindestens einmal ausgeführt
        \end{itemize}
    \end{block}
\end{frame}

\begin{frame}{do \dots while - Beispiel}
    \begin{exampleblock}{Beispiel}
        \lstinputlisting[basicstyle=\small]{../modules/control-structures/src/do-while-example.java}
    \end{exampleblock}
\end{frame}

\begin{frame}
    \begin{center}
        \includegraphics[scale=0.28]{while-do-while.jpg} \\

        \begin{flushright}
            \begin{tiny}
                \textit{[Quelle: https://m.kd-tree.com/hash/8849bbaa45fa35480b92d4dd809e2834.jpg]}
            \end{tiny}
        \end{flushright}
    \end{center}
\end{frame}

\begin{frame}{for}
    \begin{block}{for}
        \lstinputlisting[basicstyle=\small]{../modules/control-structures/src/for.java}
    \end{block}

    \begin{block}{Hinweise}
        \begin{itemize}
            \item Initialisierung wird nur einmalig am Anfang ausgeführt
            \item Bedingung wird \textbf{vor} jedem Schleifendurchlauf geprüft
            \item Schritt wird \textbf{nach} jedem Schleifendurchlauf ausgeführt
        \end{itemize}
    \end{block}
\end{frame}

\begin{frame}{for - Beispiel}
    \begin{exampleblock}{Beispiel}
        \lstinputlisting[basicstyle=\small]{../modules/control-structures/src/for-example.java}
    \end{exampleblock}
\end{frame}

\begin{frame}{for-each (enhanced for loop)}
    \begin{block}{for-each}
        \lstinputlisting[basicstyle=\small]{../modules/control-structures/src/for-each.java}
    \end{block}

    \begin{block}{Hinweise}
        \begin{itemize}
            \item Keine Zählvariable mehr
            \item Funktioniert nur mit iterierbaren Datenstrukturen, bspw. Arrays, Listen, \dots
            \item Funktioniert mit enums über die \texttt{<Enum>[] values()} Methode
                  (z.B. \texttt{DayOfWeek.values()})
            \item Laufvariable enthält immer das aktuelle Objekt
        \end{itemize}
    \end{block}
\end{frame}

\begin{frame}{for-each - Beispiel}
    \begin{exampleblock}{Beispiel}
        \lstinputlisting[basicstyle=\small]{../modules/control-structures/src/for-each-example.java}
    \end{exampleblock}
\end{frame}

\begin{frame}{Abbruchanweisungen}
    \begin{alertblock}{Hinweise}
        \begin{itemize}
            \item Schleifen können abgebrochen werden, bevor alle Schleifendurchläufe abgearbeitet sind
            \item \texttt{break} veranlasst das sofortige Verlassen der Schleife
            \item \texttt{continue} setzt die Schleife mit dem nächsten Durchgang fort
            \item Der Lesbarkeit und Nachvollziehbarkeit willen \textbf{sehr sparsam} verwenden!
        \end{itemize}
    \end{alertblock}
\end{frame}
