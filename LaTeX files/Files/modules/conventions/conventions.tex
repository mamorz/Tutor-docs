\section{Konventionen}

\begin{frame}{Konventionen}
    \begin{block}{Sind Bezeichner aussagekräftig und konsistent gewählt?}
        \begin{itemize}
            \item Selbsterklärende, eindeutige Namen für Methoden und Variablen
            \item Verwendung einzelner Buchstaben nur als Laufvariablen (i, j, k) oder Parameter einfacher mathematischer Funktionen (x, y, z)
            \item Keine weiteren Informationen im Namen kodiert (z.B. \texttt{userAgeAsInt})
            \item Groß-/Kleinschreibung (und Unterstriche bei Konstanten) zur Trennung von Wortteilen
            \item Einheitliche Groß-/Kleinschreibung für Konstanten, Variablen, Methoden und Klassen
        \end{itemize}
    \end{block}
\end{frame}

\begin{frame}{Konventionen - Beispiele}
    \begin{alertblock}{schlecht}
        \begin{itemize}
            \item \lstinputlisting[basicstyle=\small]{../modules/conventions/src/bad-sample-1.java}
            \item \lstinputlisting[basicstyle=\small]{../modules/conventions/src/bad-sample-2.java}
        \end{itemize}
    \end{alertblock}

    \pause

    \begin{exampleblock}{besser}
        \begin{itemize}
            \item \lstinputlisting[basicstyle=\small]{../modules/conventions/src/good-sample.java}
        \end{itemize}
    \end{exampleblock}
\end{frame}

\begin{frame}{Whitespaces}
    \begin{block}{Kein Whitespace nach \dots}
        \begin{center}
            \begin{tabular}{l l l}
                \textbf{Operator} & \textbf{Name}          & \textbf{Beispiel}     \\
                \toprule
                \texttt{\~}       & Bitweises Komplement   & \texttt{\~{}flag}     \\
                \texttt{!}        & Logisches Komplement   & \texttt{!splittable}  \\
                \texttt{++}       & Prefix-Inkrementierung & \texttt{++i;}         \\
                \texttt{-{}-}     & Prefix-Dekrementierung & \texttt{-{}-i;}       \\
                \texttt{.}        & Punkt                  & \texttt{Math.abs(5);} \\
                \texttt{-}        & Unäres Minus           & \texttt{-5}           \\
                \texttt{+}        & Unäres Plus            & \texttt{+4}
            \end{tabular}
        \end{center}
    \end{block}
\end{frame}

\begin{frame}{Whitespaces II}
    \begin{block}{Whitespace um \dots}
        \begin{center}
            \begin{tabular}{l l l}
                \textbf{Operator}                        & \textbf{Name}           \\
                \toprule
                \texttt{=}                               & Zuweisung               \\
                \texttt{+, -, *, /, |, \&, \&\&, ||, \^} & Binäre Operatoren       \\
                \texttt{+=, -=, *=, /=, \dots}           & Kombinierte Zuweisung   \\
                \texttt{,}                               & Kommas (z.B. Parameter) \\
                \texttt{==, !=, >=, <=, \dots}           & Vergleiche              \\
                \texttt{if, else, for, while, \dots}     & Keywords
            \end{tabular}
        \end{center}

        \vspace{1em}
        Und mehr\dots
    \end{block}
\end{frame}

\begin{frame}{Java Code Conventions}
    \begin{block}{Warum?}
        \begin{itemize}
            \item Quelltext wird vor allem \textit{gelesen} -- und nicht nur von euch!
            \item Softwarewartung erzeugt \textasciitilde80\% der Gesamtkosten für Software
            \item Quelltext ist die einzig \textit{verlässliche} Dokumentation, da er immer aktuell ist
            \item Gut geschriebener Code ist besser verständlich als Dokumentation
            \item Software wird mit Quelltexten ausgeliefert
            \item \href{http://www.oracle.com/technetwork/java/javase/documentation/codeconvtoc-136057.html}{Java Code Conventions} (JCC) sind internationaler Standard
        \end{itemize}
    \end{block}
\end{frame}
