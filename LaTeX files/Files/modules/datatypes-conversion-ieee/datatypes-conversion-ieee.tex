\graphicspath{{../modules/datatypes-conversion-ieee/img/}}

\section{Datentypen}

\subsection{Primitive Datentypen}

\begin{frame}{Primitive Datentypen}
    \begin{block}{Primitive / Elementare Datentypen in Java}
        \begin{footnotesize}
            \begin{center}
                \begin{tabular}{cccc}
                    \textbf{Typ} & \textbf{Erklärung} & \textbf{Wertebereich}                      & \textbf{Beispielwerte}                          \\
                    \midrule [1pt]
                    boolean      & Wahrheitswerte     & true oder false                            & true, false                                     \\
                    \cmidrule[0.5pt]{2-4}
                    char         & 16-Bit-Unicode     & 0x0000 \dots 0xffff                        & 'A', '\textbackslash n', '\textbackslash u05D0' \\
                    \cmidrule[0.5pt]{2-4}
                    byte         & 8-Bit-Integer      & $-2^7\dots2^7-1$                           & 12                                              \\
                    \cmidrule[0.5pt]{2-4}
                    short        & 16-Bit-Integer     & $-2^{15}\dots2^{15}-1$                     & 12                                              \\
                    \cmidrule[0.5pt]{2-4}
                    int          & 32-Bit-Integer     & $-2^{31}\dots2^{31}-1$                     & 12                                              \\
                    \cmidrule[0.5pt]{2-4}
                    long         & 64-Bit-Integer     & $-2^{63}\dots2^{63}-1$                     & 12L, 14L                                        \\
                    \cmidrule[0.5pt]{2-4}
                    float        & 32-Bit-Gleitk.     & 1,40239846E-45f\dots                       & 9.81F, 0.3E-8F, 2f                              \\
                                 &                    & 3,40282347E+38f                            &                                                 \\
                    \cmidrule[0.5pt]{2-4}
                    double       & 64-Bit-Gleitk.     & \scriptsize{4,94065645841246544E-324\dots} & 9.81, 3e1                                       \\
                                 &                    & \scriptsize{1,79769131486231570E+308}      &                                                 \\
                \end{tabular}
            \end{center}
        \end{footnotesize}
    \end{block}
\end{frame}

\subsection{Typkonvertierung}

\begin{frame}{Typkonvertierung}
    \begin{block}{Allgemein}
        \begin{itemize}
            \item Automatische Konvertierung: \textit{double x = 42;} \\
                  $\rightarrow$ Primitiven Datentyp erweitern
            \item Erzwungene Konvertierung: \textit{int y = (int) 47.11;} \\
                  $\rightarrow$ Primitiven Datentyp (aktiv!) einschränken \\
                  $\rightarrow$ Java schneidet hier die Nachkommastellen einfach ab
        \end{itemize}
    \end{block}
\end{frame}

\definecolor{ieeeSign}{HTML}{898ee5}
\definecolor{ieeeExponent}{HTML}{85dd88}
\definecolor{ieeeMantissa}{HTML}{ddb185}

\subsection{IEEE 754}

\setbeamercovered{transparent}

\begin{frame}{IEEE 754}
    \begin{block}{Speicherung der Zahl}
        \begin{center}
            \includegraphics[scale=0.333]{ieee754-data-layout.jpg}
        \end{center}
    \end{block}

    \pause

    \begin{block}{Float und Double}
        \begin{center}
            \begin{tabular}{l r r}
                {}                                 & Float (Bits) & Double (Bits) \\
                \toprule
                \textcolor{ieeeSign}{Vorzeichen}   & 1            & 1             \\
                \textcolor{ieeeExponent}{Exponent} & 8            & 11            \\
                \textcolor{ieeeMantissa}{Mantisse} & 23           & 52            \\
                \bottomrule
            \end{tabular}
        \end{center}
    \end{block}

    \pause

    \begin{block}{Umrechung ins Dezimalsystem}
        \[
            \textcolor{ieeeSign}{\pm}
            \cdot
            2^{\text{\textcolor{ieeeExponent}{Exponent}}}
            \cdot
            \text{\textcolor{ieeeMantissa}{Mantisse}}
        \]
    \end{block}
\end{frame}

\setbeamercovered{invisible}

\begin{frame}[fragile]{IEEE 754 - Wissenswertes}
    \begin{block}{Fehlerquellen}
        \begin{itemize}
            \item Zahlen sind nur \textbf{endlich} genau $\Rightarrow$ Rundungsfehler
            \item Prüfen auf Gleichheit („\texttt{==}“) ist daher oft nicht zielführend. Besser? \\

                  \pause

                  Epsilonvergleich \\

                  \begin{lstlisting}[breaklines=true]
// 0.00001 ist ein passendes Epsilon
float epsilon = 0.00001;
boolean equal = @@cMath@@.abs(x - y) < epsilon;
                \end{lstlisting}
        \end{itemize}
    \end{block}
\end{frame}

\begin{frame}{IEEE754 - Besondere Werte}
    \begin{block}{Besondere Werte}
        \begin{minipage}[t]{0.7\textwidth}
            \lstinputlisting[basicstyle=\footnotesize]{../modules/datatypes-conversion-ieee/src/special-values.txt}
        \end{minipage}% necessary percent sign for side-by-side content
        \begin{minipage}[t]{0.3\textwidth}
            \lstinputlisting[basicstyle=\footnotesize]{../modules/datatypes-conversion-ieee/src/special-values-solution.txt}
        \end{minipage}
    \end{block}
\end{frame}

\begin{frame}{IEEE754 - Hilfe und Tools}
    \begin{block}{Nützliche Links}
        \begin{itemize}
            \item \href{https://float.exposed}{https://float.exposed} \\
                  \includegraphics[scale=0.15]{ieee754-float-exposed.jpg}
            \item \href{https://www.h-schmidt.net/FloatConverter/IEEE754.html}{https://www.h-schmidt.net/FloatConverter/IEEE754.html} \\
                  \includegraphics[scale=0.15]{ieee754-float-h-schmidt.jpg}
        \end{itemize}
    \end{block}
\end{frame}
