\section{Datentypen}

\begin{frame}{Datentypen}
    \begin{block}{Typen}
        \begin{itemize}
            \item Was sind die Unterschiede zwischen $\mathbb{N}, \mathbb{R}, \mathbb{Z}$, \dots ?

                  \pause

            \item Ein \frqq Typ\flqq\ bezeichnet eine Menge \frqq gleichartiger\flqq\ Werte
            \item Er legt die möglichen Werte fest, die Variablen, Attribute, Funktionen, \dots annehmen können
            \item Java kennt acht elementare / primitive Datentypen
            \item Jede Klasse stellt ebenfalls einen eigenen Datentyp dar
        \end{itemize}
    \end{block}
\end{frame}

\subsection{Primitive Datentypen}

\begin{frame}{Primitive Datentypen}
    \begin{block}{Primitive / Elementare Datentypen in Java}
        \begin{footnotesize}
            \begin{center}
                \begin{tabular}{cccc}
                    \textbf{Typ} & \textbf{Erklärung} & \textbf{Wertebereich}                      & \textbf{Beispielwerte}                          \\
                    \midrule [1pt]
                    boolean      & Wahrheitswerte     & true oder false                            & true, false                                     \\
                    \cmidrule[0.5pt]{2-4}
                    char         & 16-Bit-Unicode     & 0x0000 \dots 0xffff                        & 'A', '\textbackslash n', '\textbackslash u05D0' \\
                    \cmidrule[0.5pt]{2-4}
                    byte         & 8-Bit-Integer      & $-2^7\dots2^7-1$                           & 12                                              \\
                    \cmidrule[0.5pt]{2-4}
                    short        & 16-Bit-Integer     & $-2^{15}\dots2^{15}-1$                     & 12                                              \\
                    \cmidrule[0.5pt]{2-4}
                    int          & 32-Bit-Integer     & $-2^{31}\dots2^{31}-1$                     & 12                                              \\
                    \cmidrule[0.5pt]{2-4}
                    long         & 64-Bit-Integer     & $-2^{63}\dots2^{63}-1$                     & 12L, 14L                                        \\
                    \cmidrule[0.5pt]{2-4}
                    float        & 32-Bit-Gleitk.     & 1,40239846E-45f\dots                       & 9.81F, 0.3E-8F, 2f                              \\
                                 &                    & 3,40282347E+38f                            &                                                 \\
                    \cmidrule[0.5pt]{2-4}
                    double       & 64-Bit-Gleitk.     & \scriptsize{4,94065645841246544E-324\dots} & 9.81, 3e1                                       \\
                                 &                    & \scriptsize{1,79769131486231570E+308}      &                                                 \\
                \end{tabular}
            \end{center}
        \end{footnotesize}
    \end{block}
\end{frame}

\begin{frame}{Datentypen}
    \begin{block}{Zeichenketten}
        \begin{itemize}
            \item Ideen?

                  \pause

            \item Zusammensetzung aus \textit{char} mittels \textit{String}

                  \pause

            \item Ist kein primitiver Datentyp. Aber woran erkennt man das?

                  \pause

            \item Genau! Beginnt mit einem Großbuchstaben!
            \item Ausserdem sind ein paar Besonderheiten z.B. beim Vergleichen zu beachten
        \end{itemize}
    \end{block}
\end{frame}
