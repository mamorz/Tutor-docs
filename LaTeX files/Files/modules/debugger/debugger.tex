\section{Debugging}

\begin{frame}{Debugging}
    \setbeamercovered{transparent}
    \begin{block}{Was ist das?}
        Debugging ist der Prozess des Identifizierens und Illiminierens
        Von Fehlern (Bugs) im Code.
    \end{block}

    \pause

    \begin{block}{Wie geht das?}
        \begin{itemize}
            \item Gute Frage...
            \item Ist immer sehr komplex und eigentlich das zeitintensivste am Entwicklungsprozess
            \item Zum Glück gibt es ein paar standartisierte Tools und Verfahren, die einem helfen können
        \end{itemize}
    \end{block}
\end{frame}

\begin{frame}{Debugging}
    \setbeamercovered{invisible}
    \begin{block}{Tools \& Verfahren}
        \begin{itemize}
            \item Debugger in der IDE
            \aitem Benutzt den! Der kann sehr hilfreich sein! 
            \pause
            \item Wilde im code Verstreute print statements mit mehr oder weniger hilfreichen Ausgaben
            \aitem Meistens unübersichtlicher als ein Debugger
            \aitem Man vergisst gerne mal Ausgaben wieder raus zu nehmen
            \aitem Kann für kleine Sachen trotzdem sehr praktisch sein
            \pause
            \item Rubber ducky debugging
            \aitem Mit einer Quietscheente oder anderen Dingen/Tieren/Menschen über den Code reden während man versucht den Fehler zu finden
            \aitem Hab Ich noch nie ausprobiert, wird aber anscheinend tatsählich gemacht 
        \end{itemize}
    \end{block}
\end{frame}

\begin{frame}
    \begin{center}
        \huge Kurze Demo zum Debugger...
    \end{center}
\end{frame}