\section{Abzüge}

\begin{frame}{Abzüge in den Abschlussaufgaben}
    \begin{block}{Teil 1}
        \begin{itemize}
            \item Verwendung von Wildcards bei Importanweisung
            \item Nicht den geeignetsten Schleifentyp gewählt, z.B. While statt For oder For statt For-Each
            \item Keine Variabilität in der implementierten Logik, z.B. viele Parameter hartkodiert (Magic Numbers)
            \item Code enthält auskommentierte Quelltext-Schnipsel, TODOs, oder unangemessene Kommentare
            \item JavaDoc ist leer oder beschreibt nur triviales, insbesondere werden die Fehlerfälle (wann Exceptions geworfen werden) nicht erwähnt (Geerbte Methoden müssen kein JavaDoc haben)
        \end{itemize}
    \end{block}
\end{frame}

\begin{frame}{Abzüge in den Abschlussaufgaben}
    \begin{block}{Teil 2}
        \begin{itemize}
            \item Sehr komplexe Codestelle fehlt ein erklärender Kommentar oder Komplexe Codestelle sollte eigentlich durch geeignete Hilfsmethoden strukturiert werden
            \item Zu tiefe Verschachtelungstiefe die trivialerweise und sinnvoll in eine private Untermethode verpackt hätte werden können.
            \item Nicht-finale Attribute, die final sein könnten
            \item Statische Methode oder Attribut in einer Klasse sollte eigentlich Instanzmethode oder Attribut sein
            \item Falsche Sichtbarkeit: Methode oder Attribut sollte eigentlich Private sein
            \item Zugriffsmethoden für Implementierungsdetails (Kapselung verletzt), z.B. für Arrays oder List
        \end{itemize}
    \end{block}
\end{frame}

\begin{frame}{Abzüge in den Abschlussaufgaben}
    \begin{block}{Teil 3}
        \begin{itemize}
            \item Objekt-Orientierte Modellierung nicht konsequent angewandt
            \item Code-Kopien, statt gemeinsame Funktionalität in Hilfsmethode oder Oberklasse
            \item Ausgaben auf Terminal sind nicht in einer UI-Klasse verkapselt, sondern über die Domänenklassen verteilt, oder stark gemischt mit Logik
            \item Objekte werden über Strings referenziert, anstatt über typisierte Java-Referenzen
        \end{itemize}
    \end{block}
\end{frame}

\begin{frame}{Abzüge in den Abschlussaufgaben}
    \begin{block}{Teil 4}
        \begin{itemize}
            \item Exception werden für den Kontrollfluss benutzt
            \item Try/catch Blöcke sind sehr groß und umfassen nicht nur die nötigen Konstrukte
            \item IndexOutOfBoundsException oder NullPointerException fangen statt Größe zuvor zu prüfen oder auf null zu prüfen
        \end{itemize}
    \end{block}

    \begin{center}
        Und vieles mehr \dots
    \end{center}
\end{frame}
