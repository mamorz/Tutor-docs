\section{enum}

\begin{frame}{\textit{enum}}
    \begin{block}{enum}
        \begin{itemize}
            \item Mit Datentyp \textit{enum} erstellt man eine neue Klasse und erweitert implizit \textit{java.lang.Enum}
            \item Vereinfacht Code erheblich - siehe Beispiel \dots
        \end{itemize}
    \end{block}

    \pause

    \begin{exampleblock}{Früher (vor Java 5.0)}
        \lstinputlisting[basicstyle=\small]{../modules/enum/src/old.java}
    \end{exampleblock}
\end{frame}

\begin{frame}{\textit{enum}}
    \begin{exampleblock}{Heute (ab Java 5.0)}
        \lstinputlisting[basicstyle=\small]{../modules/enum/src/new.java}
    \end{exampleblock}
\end{frame}

\begin{frame}{enum}
    \begin{block}{Wozu?}
        \begin{itemize}
            \item selectedBandMember kann immer nur gültige Werte ("JERRY", "BOBBY", "PHIL") enthalten
            \item Die zugewiesenen Ziffern entfallen und damit auch eine Fehlerquelle

                  \pause

            \item Speichert folglich \textbf{Aufzählungen}
            \item Behandeln wie eine eigene Klasse (\dots besitzt auch alle Funktionen!)
            \item Die Elemente sind Konstanten (Namenskonvention beachten)
        \end{itemize}
    \end{block}

    \pause

    \begin{block}{Weitere Ideen für einen \textit{enum}?}

        \pause

        \begin{itemize}
            \item enum Weekday \{MONDAY, TUESDAY, \dots \}
            \item enum Weather \{SUNNY, CLOUDY, \dots \}
        \end{itemize}

    \end{block}
\end{frame}
