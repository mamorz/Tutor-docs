\graphicspath{{../modules/exception/img/}}

\section{Exceptions}

\begin{frame}{Exceptions}
    \begin{huge}
        \begin{center}
            If anything can go wrong, it will. \\
            - Murphy's Law
        \end{center}
    \end{huge}
\end{frame}

\begin{frame}{Exceptions}
    \begin{exampleblock}{Beispiel - GOTO-Zeitalter}
        \begin{minipage}{0.5\textwidth}
            \lstinputlisting[basicstyle=\tiny]{../modules/exception/src/goto-1.java}
        \end{minipage}% necessary percent sign for side-by-side content
        \begin{minipage}{0.5\textwidth}
            Naja \dots

            \begin{itemize}
                \item Schleife durch goto
                \item Fehlerbehandlung am Ende
                \item Schleifenbedingung mitten in Schleifen
                \item Komplizierter Kontrollfluss
            \end{itemize}
        \end{minipage}
    \end{exampleblock}
\end{frame}

\begin{frame}{Exceptions}
    \begin{exampleblock}{Beispiel - ohne GOTO}
        \begin{minipage}{0.5\textwidth}
            \lstinputlisting[basicstyle=\tiny]{../modules/exception/src/goto-2.java}
        \end{minipage}% necessary percent sign for side-by-side content
        \begin{minipage}{0.5\textwidth}
            Naja \dots

            \begin{itemize}
                \item Schleifenbedingung mitten in Schleifen
                \item Zahllose if-Abfragen
                \item Komplizierter Kontrollfluss
                \item kaum besser als vorher
                \item gotos sind böse und man sollte sie \textbf{nie} benutzen (\href{https://homepages.cwi.nl/~storm/teaching/reader/Dijkstra68.pdf}{Goto considered harmful})
            \end{itemize}
        \end{minipage}
    \end{exampleblock}
\end{frame}

\begin{frame}{Exceptions}
    \begin{block}{Typische (lokale) Fehlerbehandlung}
        \lstinputlisting[basicstyle=\small]{../modules/exception/src/local.java}

        Probleme:

        \pause

        \begin{itemize}
            \item Vermischung von Programmlogik und Fehlerbehandlung
            \item Fehlerausgaben im Programm verstreut
            \item Keine Trennung von Algorithmus und Benutzerinteraktion
        \end{itemize}
    \end{block}
\end{frame}

\begin{frame}{Exceptions}
    \begin{block}{Exception}
        \begin{itemize}
            \item eine \textit{Ausnahme}
            \item Zur Laufzeit des Programms
            \item Zur Unterbrechung des normalen Kontrollflusses
        \end{itemize}
    \end{block}

    \pause

    \begin{exampleblock}{Verwendung einer Exception}
        \begin{itemize}
            \item Ein \textit{Problem} tritt auf
            \item Normales Fortfahren nicht möglich
            \item Lokale Reaktion darauf nicht sinnvoll/möglich
            \item Behandlung des Problems \textit{an anderer Stelle} nötig
        \end{itemize}
    \end{exampleblock}
\end{frame}

\begin{frame}{Exceptions}
    \begin{block}{Exceptions in Java}
        Ausnahme in Java

        \begin{itemize}
            \item echtes Objekt (Methoden, Attribute, \dots)
            \item Erzeugung mit \texttt{new}
            \item Auslösen mit \texttt{throw}
        \end{itemize}
    \end{block}

    \pause

    \begin{exampleblock}{Beispiel}
        \lstinputlisting[basicstyle=\small]{../modules/exception/src/throw-1.java}
    \end{exampleblock}
\end{frame}

\begin{frame}{Exceptions}
    \begin{exampleblock}{Auszug aus der Exceptionhierarchie}
        \begin{center}
            \includegraphics[scale=0.34]{hierarchy.png}
        \end{center}
    \end{exampleblock}
\end{frame}

\begin{frame}{Exceptions}
    \begin{block}{Exceptions in Java}
        \begin{itemize}
            \item Typ (Klasse) der Ausnahme/Exception
            \item Von Klasse \texttt{Exception} abgeleitet
            \item Mindestens zwei Konstruktoren: \texttt{Default} \& mit \texttt{String}-Parameter (mit zusätzlichen Informationen)
            \item Methoden: \texttt{getMessage()} \& \texttt{printStackTrace()}
        \end{itemize}
    \end{block}

    \pause

    \begin{block}{Error}
        \begin{itemize}
            \item Neben \texttt{Exception} existiert \texttt{Error}
            \item Schwerwiegende Fehler
            \item Keine sinnvolle Behandlung möglich
        \end{itemize}
    \end{block}
\end{frame}

\begin{frame}{Exceptions}
    \begin{block}{Ausnahmebehandlung in Java}
        \lstinputlisting[basicstyle=\small]{../modules/exception/src/try-catch-1.java}
    \end{block}

    \pause

    \begin{block}{Fall-through, die Zweite}
        \begin{itemize}
            \item Java ruft den \emph{\textbf{ersten}} passenden catch Block auf!
            \item Alle weiteren werden \emph{ignoriert}
        \end{itemize}
    \end{block}
\end{frame}

\begin{frame}{Exceptions}
    \begin{exampleblock}{Beispiel}
        \lstinputlisting[basicstyle=\small]{../modules/exception/src/try-catch-2.java}
    \end{exampleblock}
\end{frame}

\begin{frame}{Exceptions}
    \begin{block}{Exception Handler}
        \begin{itemize}
            \item Behandlung einer Ausnahme
            \item an einer Stelle
            \item irgendwo im Aufrufstack
            \item getrennt von normalen Programmcode
        \end{itemize}
    \end{block}

    \pause

    \begin{block}{Catch or specify}
        Jede ausgelöste geprüfte (checked) Exception muss

        \begin{itemize}
            \item behandelt (Exception Handler) oder
            \item deklariert (\texttt{throws})
        \end{itemize}

        werden.
    \end{block}
\end{frame}

\begin{frame}{Exceptions}
    \begin{block}{Deklaration von Ausnahmen}
        \begin{itemize}
            \item Deklaration im Methodenkopf: \texttt{private String readFile(String filename) \textbf{throws} IOException, FileNotFoundException \{\}}

                  \pause

            \item Aufrufer muss sich um Exception kümmern
            \item \texttt{throws} ist Teil der Signatur (Vorsicht beim Überschreiben) \\
                  Exceptions können in überschriebenen Methoden weggelassen werden, aber nicht hinzukommen.
            \item Nicht deklarationspflichtig sind \texttt{RuntimeException} \& \texttt{Error} (sowie deren Unterklassen)
        \end{itemize}
    \end{block}

    \pause

    \textit{Anmerkung: Da \texttt{IOException} Oberklasse von \texttt{FileNotFoundException} ist, müsste letzteres nicht extra deklariert werden. Dokumentationszwecke!}
\end{frame}

\begin{frame}{Exceptions}
    \begin{block}{Ort der Behandlung}
        Finden der passenden Ausnahmebehandlung:

        \begin{itemize}
            \item Suche im Aufrufstack nach umgebenden \texttt{try-catch}-Blöcken, gehe zu erstem passenden \texttt{catch}-Block
            \item Nach der Behandlung: Fortsetzung am Ende des \texttt{try-catch}-Block
        \end{itemize}
    \end{block}
\end{frame}

\begin{frame}{Exceptions}
    \begin{block}{Error}
        (Katastrophale) Probleme, die eigentlich nicht auftreten dürfen. Speicher voll, Illegaler Byte-Code, JVM-Fehler, \dots
    \end{block}

    \pause

    \begin{block}{\texttt{RuntimeException}}
        Durch \textit{fremde} Fehler erzeugte Probleme. Falsche Benutzung einer Klasse, Programmierfehler.
    \end{block}

    \pause

    \begin{block}{Geprüfte Exception (\emph{checked Exception})}
        Vorhersehbare und behandelbare Fehler. Datei nicht vorhanden, Festplatte voll, Fehler beim Parsen, \dots
    \end{block}
\end{frame}

\begin{frame}{Exceptions}
    \begin{block}{Behandlung?}
        \begin{itemize}
            \item \textbf{Error und Unterklassen:} Nein, nicht sinnvoll behandelbar
            \item \textbf{Exception:} Nein, viel zu allgemein
            \item \textbf{RuntimeException:} Prinzipiell Nein
            \item \textbf{Unterklassen dessen:} Programmierfehler beheben! (Ausnahme: \texttt{NumberFormatException})
            \item \textbf{Andere:} Ja, wenn sinnvoll behandelbar
        \end{itemize}
    \end{block}
\end{frame}

\begin{frame}{Exceptions}
    \begin{block}{Werfen?}
        \begin{itemize}
            \item \textbf{Error:} Ja, ggf. mit eigener Unterklasse
            \item \textbf{Exception:} Niemals, nur als eigene Unterklasse
            \item \textbf{RuntimeException:} Ja, eigene (semantisch passende) Unterklasse
        \end{itemize}
    \end{block}

    \pause

    \begin{exampleblock}{Beispiel}
        \lstinputlisting[basicstyle=\small]{../modules/exception/src/throw-2.java}
    \end{exampleblock}
\end{frame}

\begin{frame}{Exceptions}
    \begin{block}{Eigene Exceptions}
        \begin{itemize}
            \item Ableiten einer eigenen Unterklasse von Exception oder RuntimeException
            \item Implementierung der zwei Standard-Konstruktoren
            \item Definition einer eigenen, sinnvollen Exception-Hierarchie (bei Bedarf)
            \item Verwendung von vorhandenen Exceptions nur für dafür vorgesehene Zwecke (Javadoc anschauen)
        \end{itemize}
    \end{block}

    \begin{exampleblock}{Beispiele in der Java-API}
        \begin{itemize}
            \item \texttt{IllegalArgumentException}
            \item \texttt{IllegalStateException}
            \item \texttt{UnsupportedOperationException}
            \item \texttt{NullPointerException}
        \end{itemize}
    \end{exampleblock}
\end{frame}

\begin{frame}{Exceptions}
    \begin{block}{Verwendung}
        Exceptions sollen \dots
        \begin{itemize}
            \item zur Vereinfachung dienen
            \item die absolute Ausnahme darstellen
            \item mittels \texttt{@throws} im Javadoc beschrieben werden
            \item NICHT den normalen Kontrollfluss steuern
        \end{itemize}
    \end{block}

    \pause

    \begin{alertblock}{Böse!}
        \lstinputlisting[basicstyle=\small]{../modules/exception/src/catch-exception.java}
    \end{alertblock}
\end{frame}

\begin{frame}{Exceptions}
    \begin{alertblock}{Verboten!}
        \begin{itemize}
            \item \texttt{try}-Block um das ganze Programm
            \item Leerer \texttt{catch}-Block
            \item Explizites Fangen des Typs \texttt{Exception}
            \item Explizites Fangen des Typs \texttt{Throwable}
        \end{itemize}
    \end{alertblock}
\end{frame}

\begin{frame}{Exceptions}
    \begin{block}{Frühe Fehlererkennung}
        \begin{itemize}
            \item Kleinerer Suchbereich beim Debugging
            \item Abbruch/Fehlermeldung bei inkonsistentem Programmzustand
            \item Häufig: \texttt{null} oder negative Zahlen
            \item Defensive Programmierung
            \item Zusätzliche Abfragen im Programmtext
        \end{itemize}
    \end{block}
\end{frame}

\begin{frame}{Exceptions - Zusammenfassung}
    \begin{block}{Ausnahmen}
        \begin{itemize}
            \item werden ausgelöst (\texttt{throw}) und behandelt (\texttt{try-catch}) oder deklariert (\texttt{throws})
            \item sollen die Ausnahme bleiben
            \item trennen sauber Programmlogik und Fehlerbehandlung
        \end{itemize}
    \end{block}

    \pause

    \begin{block}{Fehlererkennung}
        \begin{itemize}
            \item so früh wie möglich
            \item defensiv
            \item mittels \texttt{assert} (gleich) oder \texttt{if} \& Exceptions
        \end{itemize}
    \end{block}
\end{frame}

\begin{frame}{Exceptions}
    \begin{block}{Aufgabe}
        Schreiben Sie eine eigene Exception \texttt{InputException}, die bei falschen Benutzereingaben geworfen werden soll. Dazu erbt sie von der Exception \texttt{IllegalArgumentException}.
    \end{block}
\end{frame}

\begin{frame}{Exceptions}
    \begin{exampleblock}{Mögliche Lösung - Teil 1}
        \lstinputlisting[basicstyle=\tiny]{../modules/exception/src/input-exception-1.java}
    \end{exampleblock}
\end{frame}

\begin{frame}{Exceptions}
    \begin{exampleblock}{Mögliche Lösung - Teil 2}
        \lstinputlisting[basicstyle=\tiny]{../modules/exception/src/input-exception-2.java}
    \end{exampleblock}
\end{frame}

\begin{frame}{Exceptions}
    \begin{block}{„Hausaufgabe“}
        \begin{itemize}
            \item geprüfte/ungeprüfte Ausnahmen
            \item \texttt{finally}
            \item Rethrows
            \item Wann Exception, wann assert?
            \item serialVersionUID
        \end{itemize}
    \end{block}
\end{frame}
