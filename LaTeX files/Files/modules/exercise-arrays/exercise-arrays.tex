\section{Aufgabe: Arrays}

\begin{frame}{Aufgabe - Arrays (cont.)}
    \begin{block}{Arrays - Vektorrechnung}
        \begin{itemize}
            \item \textbf{public static double[] sum(double[] vectorA, double[] vectorB)}, \\
                  die eine Vektoraddition auf den beiden übergebenen Vektoren durchführt und das Ergebnis als Rückgabewert hat. Achten Sie darauf, dass die beiden Vektoren hierzu dieselbe Länge haben müssen und dass Sie beim Berechnen der Summe weder vectorA noch vectorB verändern.
            \item \textbf{public static double[] scalarMult(double[] vectorA, double scalar)}, \\
                  die den Vektor vectorA mit dem Skalar scalar multipliziert und das Ergebnis als Rückgabewert hat. Achten Sie auch hier darauf, dass vectorA unverändert bleibt.
        \end{itemize}
    \end{block}
\end{frame}

\begin{frame}{Aufgabe - Arrays (cont.)}
    \begin{block}{Arrays - Matrixrechnung}
        \begin{itemize}
            \item \textbf{public static double[][] sum(double[][] matrixA, double[][] matrixB)}, \\
                  die die Matrizen matrixA und matrixB addiert und das Ergebnis als Rückgabewert hat. Beachten Sie, dass hierzu die Dimensionen der Matrizen gleich sein müssen und dass auch hier weder matrixA und matrixB verändert werden sollen.
            \item \textbf{public static double[][] multiply(double[][] matrixA, double[][] matrixB)}, \\
                  die die Matrizen matrixA und matrixB multipliziert und das Ergebnis als Rückgabewert hat. Beachten Sie auch hier die für die Multiplikation notwendigen Dimensionen der Matrizen und verändern Sie auch hier weder matrixA noch matrixB.
        \end{itemize}
    \end{block}
\end{frame}
