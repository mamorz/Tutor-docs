\section{Aufgabe: Automobil}

\begin{frame}{Jetzt seid ihr dran \dots}
    \begin{block}{Beispiel: Auto}
        \begin{itemize}
            \item Welche Klassen und Attribute benötigen wir, um ein Auto zu modellieren?
            \item Ist es sinnvoll, wirklich ALLE möglichen Autos modellieren zu können, oder verzichten wir zugunsten geringerer Komplexität auf gewisse Spezifikationen?
            \item Bitte auch entsprechend programmieren, kompilieren und ausführen \dots
        \end{itemize}
    \end{block}
\end{frame}

\begin{frame}{Aufgabe: Automobil}
    \begin{block}{Aufgabe: Automobil}
        In diesem Tutorium soll der Klassenentwurf für ein einfaches Automobil vorgenommen werden.
        \\ \ \\
        Schreiben Sie daher in Java die in den folgenden Teilaufgaben spezifizierten Klassen mit den zugehörigen Attributen.
        \\ \ \\
        Achten Sie dabei auf den Einsatz sinnvoller Datentypen, aussagekräftige Attributnamen, Einrückungen, Namenskonventionen, Dokumentation des Quelltextes.
    \end{block}
\end{frame}

\begin{frame}{Aufgabe: Automobil}
    \begin{block}{Karosserie}
        Die \textbf{Karosserie} eines Autos hat jeweils eine \textit{Farbe}, ein \textit{Gewicht in kg} und eine bestimmte \textit{Anzahl an Sitzplätzen}.
    \end{block}

    \begin{block}{Motor}
        Ein \textbf{Motor} wird dabei beschrieben durch die \textit{Anzahl der Zylinder},
        eine \textit{Leistung in kW}, ein \textit{Gewicht in kg} und einen \textit{Hubraum in Litern}.
    \end{block}

    \begin{block}{Auto}
        Ein \textbf{Auto} hat jeweils ein \textit{Baujahr}, einen \textit{Hersteller} und ist von einem bestimmten \textit{Typ}. Zusätzlich gehört zu einem Auto grundsätzlich eine \textit{Karosserie} und ein \textit{Motor}. Verwenden Sie hierzu die obigen Klassen.
    \end{block}

    Entwerfen Sie die Java-Klassen und kompilieren diese. Beheben Sie ggf. die vom Compiler bemängelten Syntaxfehler.
\end{frame}
