\section{Aufgabe: Commands}

\begin{frame}{Aufgabe: Commands}
    \begin{block}{Aufgabe}
        Schreibe eine kleines Commandsystem mit kompletter Paketstruktur, das sich mit „quit“ beendet.\\
        Es soll einen Kleinen Taschenrechner unterstützen, also folgende Befehle implementieren:
        \begin{itemize}
            \item \texttt{add <num|ans> <num|ans>} Addiert die zwei Argumente
            \item \texttt{sub <num|ans> <num|ans>} Zieht das zweite vom ersten Argument ab
            \item \texttt{abs <num|ans>} Berechnet den Betrag des Argumrntes
            \item \texttt{sqrt <num|ans>} Berechnet die Wurzel des Argumentes
            \item \texttt{ans} Gibt das letzte Ergebnis aus
            \item \texttt{undo} Macht den letzten Befehl rückgängig
        \end{itemize}
    \end{block}
\end{frame}

\begin{frame}{Aufgabe: Commands}
    \begin{block}{Info}
        \begin{itemize}
            \item \texttt{<num|ans>} Steht hier jeweils für eine Zahl oder den String ''ans'', bei dem das letzte Ergebnis verwendet werden soll
            \item Arbeitet so weit wie ihr kommt, vereinfacht die Aufgabe falls es schon spät ist
            \item Es geht nicht um die eigentliche Logik sondern darum das Entwurfsmuster anzuwenden
            \item Geht davon aus, dass alle Eingaben korrekt sind. Mehr dazu wie man Eingaben validiert gibt's im Tutorium zum parsen
        \end{itemize}  
    \end{block}
\end{frame}
