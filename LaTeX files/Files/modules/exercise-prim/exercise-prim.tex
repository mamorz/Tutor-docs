\section{Aufgaben: Prim}

\subsection{Primzahlen}

\begin{frame}{Aufgabe 1}
    \begin{block}{Primzahlen}
        \begin{itemize}
            \item Schreiben Sie in einer Klasse namens \textit{Loops} die Methoden

                  \begin{itemize}
                      \item \textit{\textbf{public static boolean isPrimeFor(int candidate)}} (mit for-Schleife)
                      \item \textit{\textbf{public static boolean isPrimeWhile(int candidate)}} (mit while-Schleife)
                      \item \textit{\textbf{public static boolean isPrimeDoWhile(int candidate)}} (mit do-Schleife)
                  \end{itemize}

                  Diese sollen als Rückgabewert den Wert true haben, falls \textit{candidate} eine Primzahl ist. Beachten Sie auch die Fälle, in denen der Wert des Parameters ganz offensichtlich keine Primzahl ist.

            \item Schreiben Sie zusätzlich eine main-Methode \textit{\textbf{public static void main(String... args)}}, die alle Zahlen zwischen -1 und einer festgelegten konstanten Zahl N auf ihre Primzahl-Eigenschaft überprüft.
        \end{itemize}
    \end{block}
\end{frame}

\subsection{Primfaktoren}

\begin{frame}{Aufgabe 2}
    \begin{block}{Primfaktoren}
        Unter einer Primfaktorzerlegung versteht man die Darstellung einer natürlichen Zahl \textit{n > 1} als Produkt aus Primzahlen. Diese Primzahlen werden Primfaktoren der Zahl \textit{n} genannt. Die Primfaktorzerlegung einer natürlichen Zahl \textit{n} ist dabei eindeutig. So ist beispielsweise die Primfaktorzerlegung der Zahl 262395 = 3 * 3 * 5 * 7 * 7 * 7 * 17.
        \\ \ \\
        Schreiben Sie eine Klasse, die per Kommandozeilenargument eine \textit{Zahl} (int) entgegen nimmt und zu dieser die Primfaktoren ausgibt. \\
        Sie können dabei davon ausgehen, dass Ihnen die Klasse \textit{Primes} eine Methode \textit{boolean isPrime(int p)} zur Verfügung steht, mit der Sie prüfen können, ob eine Zahl prim ist.
    \end{block}
\end{frame}
