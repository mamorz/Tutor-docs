\graphicspath{{../modules/exercise-recursion/img/}}

\section{Aufgaben: Rekursion}

\subsection{Aufgabe 1 - Fibonacci}

\begin{frame}{Aufgabe 1 - Fibonacci}
    Fibonacci-Folge\footnote{\url{http://de.wikipedia.org/wiki/Fibonacci-Folge}}:
    \\ \ \\
    \[
        f(n) =
        \begin{cases}
            n             & \text{ falls } n=0 \text{ oder } n=1 \\
            f(n-1)+f(n-2) & \text{ falls } n >= 2
        \end{cases}
    \]
\end{frame}

\subsection{Aufgabe 2 - Ackermann}

\begin{frame}{Aufgabe 2 - Ackermann}
    Ackermann-Funktion\footnote{\url{http://de.wikipedia.org/wiki/Ackermannfunktion}}:
    \\ \ \\
    \[
        A(m,n) =
        \begin{cases}
            n+1             & \text{ falls } m=0                    \\
            A(m-1,1)        & \text{ falls } m>0 \text{ und } n=0   \\
            A(m-1,A(m,n-1)) & \text{ falls } m>0 \text{ und } n > 0
        \end{cases}
    \]
\end{frame}

\subsection{Aufgabe 3 - Aufrufbäume}

\begin{frame}{Aufgabe 3 - Aufrufbäume}
    Zeichnen sie nun jeweils einen Aufrufbaum für:
    \\ \ \\

    \begin{itemize}
        \item f(5)
        \item A(1,2)
    \end{itemize}
\end{frame}

\begin{frame}{Aufgabe 3 - Aufrufbäume}
    \begin{block}{Aufrufbäume}
        \begin{minipage}{0.5\textwidth}
            \begin{center}
                \includegraphics[scale=0.35]{fibonacci.png}
            \end{center}
        \end{minipage}% necessary percent sign for side-by-side content
        \begin{minipage}{0.5\textwidth}
            \begin{center}
                \includegraphics[scale=0.25]{ackermann.png}
            \end{center}
        \end{minipage}
    \end{block}
\end{frame}

\begin{frame}{Beispiellösung}
    \vspace*{-0.55em}
    \lstinputlisting[basicstyle=\scriptsize]{../modules/exercise-recursion/src/solution.java}
    % Trick latex into thinking we ended properly
    \vspace*{-0.55em}
\end{frame}
