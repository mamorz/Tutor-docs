\section{Aufgabe: Stammbaum}

\begin{frame}{Aufgabe Stammbaum}
    \begin{block}{Aufgabe}
        Schreibe ein Programm, dass einen Stammbaum modelliert.\\
        Die Daten sollen dabei mit Hilfe eines Baumes organisiert werden.
        Der Baum soll Personen enthalten, die Namen, zwei Eltern, eine ID (einzigartig) und ein Alter haben.
        Das Programm soll folgende Befehle Befehle unterstützen: 
        \begin{itemize}
            \item \lstinline{print} Gibt den ganzen Baum mit entsprechender Einrückung aus.
            \item \lstinline{quit} Beendet das Programm (\lstinline{System::exit()} ist verboten :) )
            \item \lstinline{insert <name> <motherID> <fatherID> <age>} Fügt eine Person an der entsprechenden Stelle im Stammbaum ein.
            \item \lstinline{info <personID>} Gibt alle Informationen der entsprechenden Person aus.
            \item Wenn ihr euch langweilt: \lstinline{delete <personID>} Löscht die Person und alle ihre Nachfahren.
        \end{itemize}
        Geht davon aus, dass die Eingaben zu den Befehlen korrekt sind.
    \end{block}
\end{frame}