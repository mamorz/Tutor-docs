\section{Übungsblatt 2}

\begin{frame}{Übungsblatt 2 - Besprechung}
    \begin{itemize}
        \item Je Klasse eigene Datei
        \item Enum nicht nur anlegen, auch das passende Attribut
        \item Keine inline enums
        \item Objektorientierung! (Klassen als Datentyp nutzen, \dots)
        \item Datenkapselung (\texttt{private} \dots)
        \item Wiederverwendung eigener Methoden (Vermeidung von Code-Duplizierung)
        \item Utility-Klassen: final class, privaten Konstruktor und die Methoden/Attribute static. Und \textbf{keine Instanzen} dessen!
        \item \texttt{if(var == true)} $\Leftrightarrow$ \texttt{if(var)}
        \item \texttt{equals()} verwenden
        \item Sinnvolle Bezeichner \& Kommentare
        \item Checkstyle
    \end{itemize}
\end{frame}

\begin{frame}[fragile]{Übungsblatt 2 - Besprechung}
    \begin{itemize}
        \item Falscher Schleifentyp (while statt for, for statt for-each verwendet)
        \item Wildcard-Imports sind nicht erlaubt
        \item Auskommentierten Quelltext nicht abgeben
        \item \href{https://en.wikipedia.org/wiki/Magic_number_(programming)}{Magic Numbers}
        \item Vergleichsrichtung: \texttt{"text".equals(var)}
        \item Bitte kein \texttt{while(true)}
        \item Kompilieren, Ausführen, \textbf{TESTEN}
        \item Im Artemis nachschauen ob eswirklich kompiliert hat!
    \end{itemize}
\end{frame}

\subsection{Kleiner Einschub}

\begin{frame}[fragile]{Weitere Anmerkungen}
    \begin{itemize}
        \item Der Name einer Variablen hat nichts mit der Objektidentität zu tun: \\

              \begin{lstlisting}
FireTruck truck = new FireTruck();
FireTruck anotherTruck = truck;
// Beide Variablen zeigen auf den GLEICHEN Truck
truck == anotherTruck // true
              \end{lstlisting}

              \pause

        \item Eine Variable ist nicht notwendig, um ein Objekt zu erstellen

              \begin{lstlisting}
// Erzeugt einen FireTruck und ruft dessen Konstruktor auf
new FireTruck();
      \end{lstlisting}
    \end{itemize}
\end{frame}

\begin{frame}{Immutable Objects}
    \begin{itemize}
        \item Ein „Immutable Object“ ist ein Objekt bei dem der Zustand lediglich einmal am Anfang gesetzt wird und danach nie wieder verändert werden kann. \\
              Bsp.: alle Strings
        \item „Immutable Objects“ sind z.B. sehr hilfreich wenn die entsprechenden Objekte kopiert werden müssen, da dann lediglich die Referenz kopiert werden muss.
        \item Sie enthalten oftmals nur \emph{Daten} $\implies$ \texttt{equals}
              (\texttt{hashCode}) überschreiben lohnt sich wahrscheinlich
        \item Siehe \href{https://en.wikipedia.org/wiki/Immutable_object}{Wikipedia - Immutable Object}
    \end{itemize}
\end{frame}
