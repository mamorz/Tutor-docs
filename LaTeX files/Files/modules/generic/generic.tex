\section{Generics}

\begin{frame}{Generics}
    \begin{block}{Die Idee dahinter}
        Nur einmalige Realisierung eines Programmierkonzepts (z.B. Liste) und zwar unabhängig vom Typ, der den Basisdaten zugrunde liegt.
    \end{block}

    \begin{block}{Merke}
        Generics abstrahieren vom zugrunde liegenden Basistyp. (Interface abstrahiert von der Implementierung.)
    \end{block}
\end{frame}

\begin{frame}{Generics - Zum Nachdenken \dots}
    \begin{block}{Problem bis Java 5 \dots}
        Datenstrukturen waren prinzipiell offen für jeden Typ.

        \begin{itemize}
            \item beim Speichern wurden Objekte vom allg. Typ 'Object' entgegengenommen
            \item Rückgabewerte waren ebenfalls vom Typ 'Object'
        \end{itemize}

        Eine Liste soll aber z.B. nur T-Objekte und keine X-, Y- oder Z-Objekte enthalten. Das kann mit dem allg. Typ 'Object' jedoch nicht verhindert werden.
    \end{block}
\end{frame}

\begin{frame}[fragile]{Generics - Zum Nachdenken II \dots}
    \begin{exampleblock}{Beispiel}
        \lstinputlisting[basicstyle=\small]{../modules/generic/src/generic-1.java}

        Der 'cast' in Zeile 3 ist notwendig, da Iterator nur Daten vom Typ 'Object' liefert.

        \pause

        \begin{itemize}
            \item Warum kann hier ein Runtime Error auftreten, falls der Entwickler mit Integer falsch liegt?

                  \pause

            \item Wie könnte man das Problem vermeiden?

                  \pause

            \item Gefährlich, falls andere sie modifizieren können:
                  \begin{lstlisting}[showstringspaces=false]
List myIntList = new ArrayList();
myIntList.add("Evil Incarnate, Inc."); // Schade :(
                \end{lstlisting}
        \end{itemize}
    \end{exampleblock}
\end{frame}

\begin{frame}{Generics - Zum Nachdenken III \dots}
    \begin{exampleblock}{Beispiel zur Vermeidung}
        \lstinputlisting[basicstyle=\small]{../modules/generic/src/generic-2.java}
    \end{exampleblock}

    \pause

    \begin{block}{Seit Java 5 \dots}
        Mit Java 5 nutzt die Collection-API sehr intensiv Generics. Daraus folgt bessere Typsicherheit (nur spezielle Objekte in der Datenstruktur), da angegeben werden kann, welche Typen in der Liste erlaubt sind
    \end{block}

    \pause

    \begin{exampleblock}{Beispiel}
        Was ist hier erlaubt

        \lstinputlisting[basicstyle=\small]{../modules/generic/src/generic-3.java}
    \end{exampleblock}
\end{frame}

\begin{frame}{Generics - Typeinschränkung}
    \begin{block}{Typeinschränkung möglich!}
        Es ist möglich, den Typ des Parameters einzuschränken! \\

        \lstinputlisting[basicstyle=\small]{../modules/generic/src/generic-8.java}
    \end{block}
\end{frame}

\begin{frame}[fragile]{Generics und Vererbung}
    \begin{block}{Subtypen}
        \begin{itemize}
            \item Ist dies möglich:

                  \begin{lstlisting}
List<@@cDog@@> dogs = new ArrayList<>();
// Liste von Hunden ist-ein Liste von Tieren
List<@@cAnimal@@> animals = dogs;
                \end{lstlisting}

                  \pause

            \item \textbf{Nein!} Warum nicht?

                  \pause

                  \begin{lstlisting}
List<@@cDog@@> dogs = new ArrayList<>();
List<@@cAnimal@@> animals = dogs;
animals.add(new @@cCat@@()); // Autsch :(
                \end{lstlisting}

            \item Was ist mit Arrays?

                  \pause

                  Da geht das! \textbf{Aufpassen!}
        \end{itemize}
    \end{block}
\end{frame}

\begin{frame}{Generics - Zusammenfassung}
    \begin{block}{Syntax}
        \lstinputlisting[basicstyle=\small]{../modules/generic/src/generic-4.java}
        \lstinputlisting[basicstyle=\small]{../modules/generic/src/generic-5.java}
    \end{block}

    \begin{exampleblock}{Beispiel}
        \begin{itemize}
            \item Klassendeklaration:
                  \lstinputlisting[basicstyle=\small]{../modules/generic/src/generic-6.java}

                  \pause

            \item Verwendung:
                  \lstinputlisting[basicstyle=\small]{../modules/generic/src/generic-7.java}
        \end{itemize}
    \end{exampleblock}
\end{frame}

\begin{frame}{Generics - Zusammenfassung}
    \begin{alertblock}{Wichtig!}
        Die Typ-Parameter dürfen NUR mit Referenz-Datentypen instantiiert werden!
        \\ \ \\
        Primitive Datentypen sind also NICHT erlaubt!
    \end{alertblock}
\end{frame}
