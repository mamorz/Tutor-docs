\graphicspath{{../modules/git/images/}}


\section{git}
\subsection{Was ist git?}
    \begin{frame}{Was ist git?}
        \begin{itemize}
            \item Versionskontrollsystem um gemeinsam dezentral Softwareprojekte zu entwickeln
            \aitem Zu viel für dieses Tutorium, ihr braucht es nur um Aufgaben abzugeben 
        \end{itemize}
    \end{frame}

    \begin{frame}
        \begin{center}
            \includegraphics[width=450px]{git_lifecycle.png}    
        \end{center}
    \end{frame}

\subsection{Wichtige Befehle I}
    \begin{frame}{Wichtige Befehle I}
        \setbeamercovered{transparent}
        
        \begin{block}{\textit{git clone}}
            \begin{itemize}
                \item Läd das repository vom Server herunter 
                \item Benutzen um das Repository zu holen wenn man es noch nicht hat
            \end{itemize}
        \end{block}

        \pause

        \begin{block}{\textit{git status}}
            \begin{itemize}
                \item Hilfe mein repo brennt!
                \aitem git status zeigt auch an was los ist
            \end{itemize}
        \end{block}   

        \pause

        \begin{block}{\textit{git add}}
            \begin{itemize}
                \item Verschiebt eine Datei von der untracked in die staged area
                \item Wenn ihr die datei nicht added wird sie von git ignoriert
                \aitem Fügt alle Dateien auch wirklich hinzu!
            \end{itemize}
        \end{block}          
    \end{frame}

\subsection{Wichtige Befehle II}    
    \begin{frame}{Wichtige Befehle II}
        \setbeamercovered{transparent}

        \begin{block}{\textit{git commit}}
            \begin{itemize}
                \item Bestätigt die gemachten Änderungen an der Datei
                \item Stellt quasi einen snapshot dar, zu dem man zurückkehren kann
                \item Wichtig: Aussagekräftige commit mesages schreiben, die beschreiben was ihr geändert habt
            \end{itemize}
        \end{block}

        \pause

        \begin{block}{\textit{git push}}
            \begin{itemize}
                \aitem lädt die Änderungen auf den Server hoch
            \end{itemize}
        \end{block}
    \end{frame}

    \begin{frame}{ }
        \begin{center}
            \Huge{Livebeispiel anhand von Artemis!}
        \end{center}
        
    \end{frame}