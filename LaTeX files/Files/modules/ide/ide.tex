\section{IDE}

\begin{frame}{IDE}
    \begin{block}{Was ist eine IDE?}
        \begin{itemize}
            \item IDE steht für \textbf{I}ntegrated \textbf{D}evelopment \textbf{E}nviroment
            \item Vereinfacht den Entwicklungsprozess, indem sie tools bereitstellt 
            \begin{itemize}
                \item Debugger
                \item Git Integration
                \item Quelldateienverwaltung
                \item ...
            \end{itemize}
            \item Können sehr komplexe Dinge, die man fast alle nie braucht
        \end{itemize}
    \end{block}

    \pause

    \begin{block}{Welche IDE?}
        \begin{itemize}
            \item Es gibt viele viele verschiedene IDEs...
            \aitem z.B. eclipse, IntelliJ, NetBeans...
            \item Die Vorlesung nutzt \href{https://www.eclise.org/downloads/}{eclipse}
            \item \href{https://www.jetbrains.com/idea/}{IntelliJ} ist auch sehr gut (als Studierende bekommt ihr die ultimate edition kostenlos)
        \end{itemize}
    \end{block}
\end{frame}