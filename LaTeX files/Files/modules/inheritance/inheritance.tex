\section{Vererbung}

\setbeamercovered{transparent}

\begin{frame}{Warnung!}
    \begin{alertblock}{Warning - Warning - Warning}
        Dieses Tutorium beinhaltet heute teils (sehr) schw\dots herausfordernde Inhalte!
    \end{alertblock}
\end{frame}

\setbeamercovered{invisible}

\begin{frame}{Was ist Vererbung? - Motivation}
    \begin{block}{Biologie \dots}
        \begin{itemize}
            \item Hund, Pudel, Katze, Vogel, Road Runner, Kojote \\
                  Findet ihr Oberbegriffe?

                  \pause

            \item Findet ihr gemeinsame Verhaltensweisen oder Eigenschaften?

                  \pause

            \item Unterscheiden sich die Verhaltensweisen vielleicht immer etwas?
        \end{itemize}
    \end{block}

    \pause

    \begin{block}{\dots und die Informatiker}
        \begin{itemize}
            \item Findet ihr hier vielleicht Klassen (Mit Methoden!)?

                  \pause

            \item Haben diese Klassen Gemeinsamkeiten?
        \end{itemize}
    \end{block}
\end{frame}

\begin{frame}[fragile]{Was ist Vererbung? - Motivation}
    \begin{block}{Und wo ist jetzt die Vererbung?}

        \begin{itemize}
            \item Manche Dinge kann jedes Lebewesen. Bewegen zum Beispiel. Hat man nun also eine Liste von Lebewesen, kann man sie alle zum Bewegen bringen -- ungeachtet ihrer Gattung oder Art
            \item Dieses Konzept (Gemeinsamkeiten im Verhalten oder Eigenschaften) liegt der Vererbung zu Grunde.

                  \pause

            \item \begin{lstlisting}
@@cAnimal@@ animal = new @@cDog@@();
animal.moveTo(sierra);
animal = new @@cBird@@();
animal.moveTo(farSouth);
                \end{lstlisting}

        \end{itemize}

    \end{block}
\end{frame}

\begin{frame}{Vererbung}
    \vspace*{-4.5pt}

    \begin{block}{Wozu Vererbung?}
        Wir verwenden Vererbung für:

        \begin{itemize}
            \item Spezialisierungen (\texttt{moveTo} in einem Vogel macht nicht das gleiche wie in einem Hund!)

                  \pause

            \item Code-Wiederverwendung (\texttt{getAge} ist vielleicht überall gleich) \\

                  \begin{itemize}
                      \item Weniger Code-Redundanz $\rightarrow$ erhöhte Wartbarkeit
                  \end{itemize}
        \end{itemize}
    \end{block}
\end{frame}

\begin{frame}{Vererbung}
    \begin{block}{Merke \dots}
        \begin{itemize}
            \item Modelliert \textit{ist-ein-Beziehungen} \\
                  (ein Student ist ein Mensch, ein Mensch ist ein Säugetier, ein Säugetier ist ein Lebewesen)

                  \pause

            \item Kennzeichnung mittels „\texttt{extends}“

                  \pause

            \item Keine Mehrfachvererbung in Java (nur eine Oberklasse je Klasse)

                  \pause

            \item Jede Klasse erbt implizit von Object (equals, toString, hashcode, \dots) \\
                  Dies ist wie das „Lebewesen“ in unserem Beispiel. Alles ist ein Lebewesen und in Java ist alles ein Objekt.

                  \pause

            \item „\texttt{final}“ verhindert Vererbung von der Klasse

                \pause

            \item Grundsätzlich gilt das \textbf{Liskov'sche Substitutionsprinzip}
            \aitem Jede Instanz einer Subklasse sollte auch als Instanz einer Superklasse einsetzbar sein\\
            Beispiel: Wenn nur ein Objekt der Klasse \texttt{Animal} gefordert ist, sollte insbesondere auch ein Objekt der Klasse \texttt{Dog} in dem Kontext funktionieren, da ein \texttt{Dog} ja ein \texttt{Animal} ist. 
        \end{itemize}
    \end{block}
\end{frame}

\subsection{Überschreiben}

\begin{frame}[fragile]{Vererbung - Überschreiben}
    \begin{block}{Gut zu wissen \dots}
        \begin{itemize}
            \item Angenommen, der Standard für \texttt{moveTo} für Lebewesen ist Laufen.
            \item Wie ändern wir \texttt{moveTo} in unserem Vogel, sodass er fliegt? Überschreiben!

                  \begin{lstlisting}
class @@cBird@@ {
    // Hilfs-Marker fuer den Compiler
    @@c@Override@@
    // gleicher Name, Parameter und return type
    public void moveTo(Location location) {
        // Was es bei einem Aufruf machen soll
        flyTo(location);
    }
}
                \end{lstlisting}

            \item Damit wird nun bei jedem Aufruf von \texttt{moveTo} \texttt{flyTo} aufgerufen
        \end{itemize}
    \end{block}
\end{frame}

\begin{frame}{Vererbung - Überschreiben}
    \begin{block}{Gut zu wissen \dots}
        \begin{itemize}
            \item Will man Methodenverhalten von Oberklassen ändern, so kann man diese Methoden überschreiben

                  \pause

            \item Methodenname muss identisch zu dem in der Oberklasse sein

                  \pause

            \item Parameter müssen identisch zu denen in der Oberklasse sein

                  \pause

            \item Rückgabewert muss identisch zum Rückgabewert der Methode in der Oberklasse sein (gleich oder Subtyp)

                  \pause

            \item 'final' Methoden können nicht überschrieben werden

                  \pause

            \item \textit{@Override} stellt sicher, dass Überschreiben tatsächlich stattfindet (sonst Compilerfehler)

                  \begin{itemize}
                      \item Überschriebene Methoden am besten IMMER mit \textit{@Override} kennzeichnen
                  \end{itemize}

        \end{itemize}
    \end{block}
\end{frame}

\begin{frame}{Vererbung - Überschreiben - Beispiele}
    \vspace*{-3.4pt}

    \lstinputlisting[basicstyle=\footnotesize]{../modules/inheritance/src/override.java}

    \vspace*{-3.4pt}
\end{frame}

\subsection{Überladen}

\begin{frame}{Überladen}
    \begin{block}{Gut zu wissen \dots}
        \begin{itemize}
            \item Es kann mehrere Methoden mit gleichem Namen geben. Diese müssen sich aber in Art und/oder Anzahl ihrer Parameter unterscheiden

                  \pause

                  \begin{itemize}
                      \item void doSomething(int a)
                      \item void doSomething(int a, int b)
                      \item void doSomething(int a, String b)
                      \item void doSomething(int a, int b, double c)
                      \item void doSomething(int a, long b)
                  \end{itemize}

            \item Dies gilt ebenso für Konstruktoren

                  \pause

            \item Dies hat \textbf{NICHTS} mit Vererbung zu tun!
        \end{itemize}
    \end{block}
\end{frame}

\begin{frame}{Überladen - Beispiele}
    \lstinputlisting[basicstyle=\footnotesize]{../modules/inheritance/src/overload.java}
\end{frame}

\setbeamercovered{invisible}

\begin{frame}{Überladen - Beispiele}
    \begin{block}{Beispiele}
        \begin{itemize}
            \item Was gibt \texttt{anstrengend(2)} aus?

                  \pause

            \item Was gibt \texttt{anstrengend("Bobby Tables")} aus?

                  \pause

            \item Was gibt \texttt{anstrengend(2.5)} aus?

                  \pause

            \item Was gibt \texttt{anstrengend(2.5f)} aus?

                  \pause

            \item Was gibt \texttt{anstrengend(2L)} aus?
        \end{itemize}
    \end{block}
\end{frame}

\setbeamercovered{transparent}
