\graphicspath{{../modules/java-setup/img/}}

\section{Java}

\subsection{Installation}

\begin{frame}{Installation Java}
    \begin{block}{JDK oder OpenJDK}
        Beides ist für unsere Zwecke OK. OpenJDK ist seit 2006 offizieller freier Nachfolger des JDK.
    \end{block}

    \begin{block}{Mac OS}
        Bereits mit Betriebssystem installiert
    \end{block}

    \begin{block}{Linux}
        Installation über Paketverwaltung
    \end{block}

    \begin{block}{Windows}
        \begin{itemize}
            \item Download \href{https://adoptopenjdk.net}{OpenJDK} (Alternativ: \href{https://www.oracle.com/technetwork/java/javase/downloads/jdk11-downloads-5066655.html}{Java SE Development Kit 11})
            \item evtl. Umgebungsvariable (PATH) ergänzen - oder immer lange Pfade tippen \\
                  \textit{z.B C:\textbackslash Program Files\textbackslash Java\textbackslash jdk-11\textbackslash bin}
        \end{itemize}
    \end{block}
\end{frame}

\subsection{JDK}

\begin{frame}{JDK}
    \setbeamercovered{transparent}
    \begin{block}{Der Java Compiler - \textit{javac}}
        \begin{itemize}
            \item Übersetzt den Java-Sourcecode (*.java) in Java-Bytecode (*.class)
            \item Fehler-Report bei syntaktisch inkorrekten Java-Klassen
            \item Kommandozeilenbefehl: \textit{javac Klassenname.java}
        \end{itemize}
    \end{block}

    \pause

    \begin{block}{Der Java Loader - \textit{java}}
        \begin{itemize}
            \item Führt den durch javac generierten Bytecode aus
            \item Benötigt sog. \textit{main}-Methode
            \item Kommandozeilenbefehl: \textit{java Klassenname}
        \end{itemize}
    \end{block}

    \pause

    \begin{block}{Der Java Dokumentation Generator - \textit{javadoc}}
        Aus den Kommentaren im Java-Sourcecode wird automatisch eine Dokumentation generiert.
    \end{block}
\end{frame}

\begin{frame}{Java-Shell (JShell)}
    \begin{block}{Die Java-Shell - \textit{jshell}}
        \begin{itemize}
            \item Seit Java 9
            \item Ist ein (mächtiger) Kommandointerpreter
            \item Read-Eval-Print-Loop
            \item Kommandozeilenbefehl: \textit{jshell}
            \item Gut, um schnell etwas zu testen
        \end{itemize}
    \end{block}
\end{frame}

\begin{frame}{JDK - Was passiert hier?}
    \begin{figure}
        \includegraphics[scale=0.4]{jdk-1.png}
    \end{figure}
\end{frame}

\begin{frame}{JDK - Was passiert hier?}
    \begin{figure}
        \includegraphics[scale=0.4]{jdk-2.png}
    \end{figure}
\end{frame}

\begin{frame}{JDK - Was passiert hier?}
    \begin{figure}
        \includegraphics[scale=0.4]{jdk-3.png}
    \end{figure}
\end{frame}

\subsection{Sourcecode erstellen}

\begin{frame}{Java-Sourcecode erstellen}
    \begin{block}{Editor}
        \begin{itemize}
            \item Java-Sourcecode schreibt man mit einem Editor. Zum Beispiel:
                  \begin{tabular}{ll}
                      Windows: & Notepad++, Notepad           \\
                      Linux:   & vi, vim, neovim, emacs, nano \\
                      Mac:     & TextEdit, SimpleText         \\
                  \end{tabular}

                  \pause

            \item Warum keine Programme wie Word, Writer, \dots ?
        \end{itemize}
    \end{block}

    \pause

    \begin{block}{Integrierte Entwicklungsumgebung (IDE)}
        IDEs wie \textbf{IntelliJ IDEA}, Eclipse, (Visual Studio Code,) \dots

        \pause

        \begin{itemize}
            \item können (gerne) genutzt werden
            \item werden von mir erst später unterstützt
            \item \textit{\textbf{Im eigenen Interesse: Schreibt die ersten Übungsaufgaben im Texteditor!}}
        \end{itemize}
    \end{block}
\end{frame}

\subsection{Quelltext erstellen - LIVE}

\begin{frame}{Eclipse und Java-Quelltext erstellen - LIVE}
    \begin{center}
        \begin{Huge}
            Hallo Welt!
        \end{Huge}
    \end{center}
\end{frame}
