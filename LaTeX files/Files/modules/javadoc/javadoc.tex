\graphicspath{{../modules/javadoc/img/}}

\section{Javadoc}

\begin{frame}{Javadoc - Wie sieht das aus?}
    \includegraphics[scale=0.195]{class.jpg}
\end{frame}

\begin{frame}{Javadoc - Wie sieht das aus?}
    \includegraphics[scale=0.22]{method.jpg}
\end{frame}

\begin{frame}{Javadoc - Wie sieht das aus?}
    \includegraphics[scale=0.195]{html.jpg}
\end{frame}

\begin{frame}{Javadoc - Was ist das?}
    \begin{block}{Javadoc\dots}
        \begin{itemize}
            \item \dots\ beschreibt Klassen, Methoden und Felder
            \item \dots\ ist unkompliziert zu schreiben und direkt im Code
            \item \dots\ erleichtert das Verständnis
        \end{itemize}
    \end{block}

    \begin{block}{Das \texttt{javadoc} Programm}
        \begin{itemize}
            \item Generiert HTML(5) code
            \item Seit einigen Java-Versionen sogar mit Suche!
            \item \href{https://docs.oracle.com/en/java/javase/12/docs/api/index.html}{Online Javadoc}
        \end{itemize}
    \end{block}
\end{frame}

\begin{frame}{Javadoc - Etiquette}
    \begin{block}{Regeln und Ziele}
        \begin{itemize}
            \item \textbf{EINHEITLICH} (Deutsch oder Englisch)
            \item Beschreiben das Wie und Warum
            \item Beschreiben Zusicherungen und Bedingungen der Ein- und Ausgabe
        \end{itemize}
    \end{block}
\end{frame}

\begin{frame}{Javadoc - Etiquette}
    \includegraphics[scale=0.2]{conventions.jpg}
\end{frame}
