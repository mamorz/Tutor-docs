\graphicspath{{../modules/junit/img/}}

\section{JUnit}

\begin{frame}
    \begin{block}{JUnit?}
        \begin{itemize}
            \item Ist ein Framework zum Schreiben von Tests
            \item Komponententest für viele Spachen: PyUnit, CUnit, \dots
            \item Umfassende Beschäftigung in Softwaretechnik I
        \end{itemize}
    \end{block}
\end{frame}

\begin{frame}{JUnit}
    \begin{block}{Verwendung von JUnit}
        \begin{itemize}
            \item für jede Klasse, die ihr testen möchtet, erstellt ihr eine Testklasse
            \item diese enthält Testfälle (Methoden)
            \item Testfälle werden durch Annotationen (\textit{@Test}) beschrieben
            \item das Testergebnis wird über Asserstions bestimmt
        \end{itemize}
    \end{block}
\end{frame}

\begin{frame}[fragile]{JUnit}
    \begin{exampleblock}{Beispiel}
        \begin{lstlisting}
import static org.junit.Assert.*;
import org.junit.*;
public class @@cMatrixTest@@ {
    private @@cMatrix@@ matrix;

    @Before
    public void setUp() {
        double[][] array = new double[][] {{1,2,3}, {4,5,6}};
        this.matrix = new @@cMatrix@@(array);
    }
    @Test
    public void rowsTest() {
        assertEquals(2, matrix.rows());
    }
}
        \end{lstlisting}
    \end{exampleblock}
\end{frame}

\begin{frame}{JUnit - Assertions}
    \includegraphics[scale=0.65]{assert.png}
\end{frame}

\begin{frame}{JUnit - Annotationen}
    \includegraphics[scale=0.65]{annotation.png}
\end{frame}

\begin{frame}{JUnit}
    \begin{block}{Test Driven Developement}
        (Oder auch hochladen bis Artemis glücklich ist :'D)

        \begin{center}
            \includegraphics[scale=0.18]{tdd.png} \\
        \end{center}

        \begin{flushright}
            \begin{tiny}
                \textit{[Quelle: https://marsner.com/wp-content/uploads/test-driven-development-TDD.png - TDD]}
            \end{tiny}
        \end{flushright}
    \end{block}
\end{frame}
