\section{Methoden}

\subsection{Methoden}

\begin{frame}{Methoden}
    \begin{block}{Wozu?}

        \pause

        \begin{itemize}
            \item Methoden realisieren das dynamische Verhalten von Objekten

                  \pause

            \item Methoden berechnen etwas aus dem aktuellen Zustand und liefern das Ergebnis zurück

                  \pause

            \item Methoden können auch ganz wildes Zeug machen \dots
        \end{itemize}
    \end{block}

    \begin{block}{Deklaration}
        \lstinputlisting[basicstyle=\small]{../modules/methods/src/method.java}

        \begin{itemize}
            \item Ergebnistyp: void (nichts) oder entsprechende(r) Datentyp / Klasse
            \item Parameterliste: Typ\textsubscript{1} name\textsubscript{1}, Typ\textsubscript{2} name\textsubscript{2}, \dots, Typ\textsubscript{n} name\textsubscript{n}
        \end{itemize}
    \end{block}
\end{frame}

\begin{frame}{Methoden II}
    \begin{block}{Methodenaufruf}
        \begin{itemize}
            \item \texttt{objectName.methodName(Arguments);}

                  \pause

                  \begin{itemize}
                      \item Und was, wenn Ergebnistyp != \textit{void}?
                  \end{itemize}
            \item Argumentenliste: Ausdruck\textsubscript{1}, \dots, Ausdruck\textsubscript{n}
        \end{itemize}
    \end{block}

    \begin{block}{Mehrere Methoden mit gleichem Namen}
        \begin{itemize}
            \item Es können mehrere Methoden mit dem gleichen Namen existieren.
            \item Dabei müssen sie sich aber in der Art und/oder Anzahl ihrer Parameter unterscheiden.
            \item Stichworte: Überladen, \textcolor{gray}{Überschreiben}
        \end{itemize}
    \end{block}
\end{frame}

\begin{frame}{Methoden - Beispiele}
    \begin{small}
        \begin{exampleblock}{Methoden mit gleichem Namen}
            \begin{multicols}{2}
                \begin{itemize}
                    \item void doSomething(int a, int b)
                    \item void doSomething(int a, String b)
                    \item void doSomething(int a, int b, double c)
                    \item void doSomething(int a, long b)
                \end{itemize}
            \end{multicols}
        \end{exampleblock}
    \end{small}

    \pause

    \begin{exampleblock}{toString-Methode - üblicherweise in jeder Klasse}
        \vspace*{-0.8em}
        \lstinputlisting[basicstyle=\small]{../modules/methods/src/to-string.java}
    \end{exampleblock}
\end{frame}

\subsection{main-Methode}

\begin{frame}{main-Methode}
    \begin{block}{Allgemein}

        \pause

        \begin{itemize}
            \item Einstiegspunkt in das Programm
            \item Jedes \textbf{ausführbare} Programm muss die main-Methode \textbf{genau einmal} enthalten
            \item Programm endet beim Verlassen der main-Methode
        \end{itemize}
    \end{block}

    \pause

    \begin{exampleblock}{Wie funktioniert das mit diesen „Kommandozeilenargumenten“?}

        \pause

        \begin{itemize}
            \item \texttt{public static void main(String[] args) \{ \\
                      /* Anweisungen */ \\
                      \}}
            \item \texttt{String... args} ist auch erlaubt
            \item Aufruf mit \texttt{java Programm a 1 Text etc}
            \item Achtung, die Elemente sind alle Strings!
        \end{itemize}
    \end{exampleblock}
\end{frame}

\subsection{Konstruktoren}

\begin{frame}{Konstruktoren}
    \begin{block}{Wozu?}
        Mit Konstruktoren werden \dots

        \pause

        \begin{itemize}
            \item neue Objekte erzeugt

                  \pause

            \item Anfangswerte von Objekten festgelegt

                  \pause

            \item Startzustände von Objekten bestimmt
        \end{itemize}
    \end{block}

    \pause

    \begin{block}{Merkmale}

        \pause

        \begin{itemize}
            \item Name des Konstruktors entspricht dem Namen der Klasse

                  \pause

            \item Objekte werden mit dem Operator '\textit{new}' erzeugt:

                  \pause

            \item Mehrere Konstruktoren mit gleichem Namen können existieren
                  \begin{itemize}
                      \item Dabei müssen sie sich aber in der Art und/oder Anzahl ihrer Parameter unterscheiden.
                  \end{itemize}
        \end{itemize}
    \end{block}
\end{frame}

\begin{frame}{Konstruktoren - Beispiel}
    \begin{exampleblock}{Wie könnten wir ein kleines Auto der Klasse \textit{Car} instantiieren?}

        \pause

        \texttt{Car smallCar = new Car();}
        \\ \ \\
        Es könnten natürlich auch noch Parameter mit übergeben werden:
        \\ \ \\
        \texttt{Car smallCar = new Car(``Smart'');}
    \end{exampleblock}
\end{frame}

\begin{frame}{Konstruktoren}
    \begin{block}{Konstruktoren - Wissenswertes}
        \begin{itemize}
            \item Hat eine Klasse keinen Konstruktor, wird der parameterlose \textit{default}-Konstruktor generiert.

                  \pause

            \item Schreibt ihr einen Konstruktor, wird kein default-Konstruktor erzeugt. \\

                  \pause

            \item Konstruktoren in Java können sich gegenseitig aufrufen. \\
                  \textbf{Dies MUSS an erster Stelle im aufrufenden Konstruktor geschehen!} (sonst Compiler-Fehler!)

                  \pause

            \item Aufzurufender Konstruktor wird als normale Methode angesehen und mit \textit{this} aufgerufen. \\
                  Compiler unterscheidet anhand der () automatisch zum this-Pointer.
        \end{itemize}
    \end{block}
\end{frame}

\begin{frame}{Konstruktoren}
    \begin{exampleblock}{Konstruktoren - ohne Verkettung}
        \lstinputlisting[basicstyle=\small]{../modules/methods/src/constructor-1.java}
    \end{exampleblock}
\end{frame}

\begin{frame}{Konstruktoren}
    \begin{exampleblock}{Konstruktoren - mit Verkettung}
        \lstinputlisting[basicstyle=\small]{../modules/methods/src/constructor-2.java}
    \end{exampleblock}
\end{frame}

\subsection{static}

\begin{frame}{static}
    \begin{exampleblock}{Beispiel}
        \lstinputlisting[basicstyle=\small]{../modules/methods/src/static-1.java}
    \end{exampleblock}
\end{frame}

\begin{frame}{static}
    \begin{exampleblock}{Beispiel}
        \lstinputlisting[basicstyle=\small]{../modules/methods/src/static-2.java}
    \end{exampleblock}
\end{frame}
