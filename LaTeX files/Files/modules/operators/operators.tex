\section{Operatoren}

\begin{frame}{Operatoren}
    \begin{block}{Präzedenz}
        \begin{itemize}
            \item Eine Präzedenz bestimmt welche Operation bzw. Operator zuerst ausgewertet wird
            \item Bekanntes Beispiel: Punkt vor Strich
            \item 1 ist nun die höchste Präzedenz, 11 die niedrigste
        \end{itemize}
    \end{block}
\end{frame}

\begin{frame}{Operatoren}
    \begin{block}{Arithmetische Operatoren}
        \begin{center}
            \begin{tabular}[h]{rlr}
                Operator             & Beschreibung                     & Präzedenz \\
                \midrule[1pt]
                +x, -x               & Vorzeichen (unäres Plus/Minus)   & 1         \\
                x * y, x / y, x \% y & Multiplikation, Division, Modulo & 2         \\
                x + y, x - y         & Addition, Subtraktion            & 3         \\
            \end{tabular}
        \end{center}
    \end{block}

    \pause

    \begin{block}{Vergleichsoperatoren}
        \begin{center}
            \begin{tabular}[h]{rlr}
                Operator                     & Beschreibung               & Präzedenz \\
                \midrule[1pt]
                x < y, x <= y, x > y, x >= y & Größenvergleich            & 5         \\
                x == y, x != y               & Gleichheit \& Ungleichheit & 6         \\
            \end{tabular}
        \end{center}
    \end{block}

    \pause

    \begin{center}
        Vorsicht geboten bei Vergleichen! \\
        Siehe Gleitkommazahlen und Identität von Objekten.
    \end{center}
\end{frame}

\begin{frame}[fragile]{Operatoren}
    \begin{block}{Wahrheitswerte}
        \begin{center}
            \begin{tabular}[h]{rlr}
                Operator & Beschreibung & Präzedenz \\
                \midrule[1pt]
                !x       & NOT          & 1         \\
                x \&\& y & AND          & 10        \\
                x || y   & OR           & 11        \\
            \end{tabular}
        \end{center}
    \end{block}

    \pause

    \begin{block}{Kurzauswertung}
        \begin{itemize}
            \item Die logischen Operatoren auf Wahrheitswerten (\texttt{||} und \texttt{\&\&}) können kurzschließen\\
                  $\quad \Rightarrow$ Bei \texttt{false \&\& true} wird nur die linke Seite ausgewertet
            \item Dies ist gerade mit \textit{null} checks sinnvoll:

                  \begin{lstlisting}[breaklines=true]
if(x != null && x.age > 18)
            \end{lstlisting}
        \end{itemize}
    \end{block}
\end{frame}

\begin{frame}{Operatoren}
    \begin{block}{Bitweise Operatoren}
        \begin{center}
            \begin{tabular}[h]{rlr}
                Operator          & Beschreibung               & Präzedenz \\
                \midrule[1pt]
                \textasciitilde x & Bitweises Komplement       & 1         \\
                x <{}< y          & Linksshift (arithmetisch)  & 4         \\
                x >{}> y          & Rechtsshift (arithmetisch) & 4         \\
                x >{}>{}> y       & Rechtsshift (logisch)      & 4         \\
                x \& y            & Bitweises Und              & 7         \\
                x \^{} y          & Bitweises XOR              & 8         \\
                x | y             & Bitweises Oder             & 9         \\
            \end{tabular}
        \end{center}
    \end{block}
\end{frame}

\begin{frame}[fragile]{Operatoren}
    \begin{alertblock}{Achtung!}
        \begin{itemize}
            \item „\texttt{\&}“ und „\texttt{|}“ funktionieren auch bei booleans
            \item Sie bewirken dort aber \textbf{keine} Kurzauswertung
            \item Und sie haben eine \textbf{höhere} Präzedenz

                  \begin{lstlisting}
if(x & y || z) /* is */ if( (x & y) || z)
            \end{lstlisting}
        \end{itemize}
    \end{alertblock}
\end{frame}

\begin{frame}{Operatoren}
    \begin{block}{Verkürzende Operatoren}
        \begin{center}
            \begin{tabular}[h]{rlr}
                Operator              & Beschreibung                & Äquivalenz              \\
                \midrule[1pt]
                x = y++, x = y--      & Postinkrement \& -dekrement & x = y; y = y + 1; \dots \\
                x = ++y, x = --y      & Präinkrement \& -dekrement  & y = y + 1; x = y; \dots \\
                x += y, x *= y, \dots & Kurzschreibweise            & x = x + y, \dots        \\
            \end{tabular}
        \end{center}
    \end{block}
\end{frame}
