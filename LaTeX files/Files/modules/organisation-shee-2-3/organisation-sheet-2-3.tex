\section{Organisatorisches}

\begin{frame}{Organisatorisches}
    \setbeamercovered{transparent}
    \begin{alertblock}{Übungsblatt 2 Abgabe}
        17. November 12:00 Uhr - 25. November 2021 6:00 Uhr
    \end{alertblock}

    \pause

    \begin{block}{Übungsschein}
        \begin{itemize}
            \item Anmeldung: 1. November 2021 12:00 Uhr – 20. Dezember 2021 12:00 Uhr\\
            \item Im \href{https://campus.studium.kit.edu/}{Campussystem}
            \item Meldet euch so bald wie möglich an und erspart euch Stress :)
        \end{itemize}
        
    \end{block}
\end{frame}

\begin{frame}{1. Übungsblatt}
    \begin{block}{Anmerkungen}
        \begin{itemize}
            \item Keine unnötigen Sachen ausgeben
            \item Aufgabenstellung genau lesen
            \item Nicht angeben (außer ihr seid 200\% sicher, dass ihr es könnt :D )
            \begin{itemize}
                \item Ihr dürft Sachen benutzen, die noch nicht dran kamen, aber wenn es falsch ist gibt es eben Punktabzug
                \item Nicht unnötig flexen und Punkte verlieren
            \end{itemize}
            \item Nicht abschreiben. Es funktioniert nicht.
            \item Keine unnötig komplexen Sachen machen.\\
            Wenn man etwas einfach aber gut lösen kann, ist das besser als kompliziert und gut.
            \item Ab dem zweiten Blatt ist der Checkstyle relevant!
        \end{itemize}
        
    \end{block}

\end{frame}

