\section{Pakete}

\begin{frame}{Pakete}
    \begin{block}{Allgemein}
        \begin{itemize}
            \item Pakete in Java dienen der Strukturierung des Quelltextes und sollen Namenskonflikte vermeiden
            \item Innerhalb eines Pakets ist jeder Name eine Klasse eindeutig

                  \begin{itemize}
                      \item Der gleiche Name kann jedoch in einem anderen Pakete
                            frei zur Bezeichnung einer anderen Klasse benutzt
                            werden (siehe \texttt{java.util.List} und
                            \texttt{java.awt.List})
                  \end{itemize}

                  \pause

            \item Deklaration mit \texttt{package edu.kit.informatik;} über dem Klassenbeginn
            \item Eine Klasse ist Teil eines Paketes, wenn:

                  \begin{itemize}
                      \item Sie in dem Ordner mit dem Paketnamen liegt
                      \item Sie \texttt{package <paketname>;} ganz am Anfang stehen hat
                  \end{itemize}

                  \pause

            \item Der Namen von Paketen in Java wird immer klein geschrieben!
        \end{itemize}
    \end{block}
\end{frame}

\begin{frame}{Pakete}
    \begin{block}{Import von Paketen}
        \begin{itemize}
            \item Um eine Klasse verwenden zu können, muss angegeben werden, in welchem Paket sie liegt. Hierzu gibt es zwei unterschiedliche Möglichkeiten.
            \item Die Klasse wird über ihren vollen (qualifizierten) Namen angesprochen: \\
                  \texttt{java.util.Date d = new java.util.Date();}
            \item Am Anfang des Programms werden die gewünschten Klassen mit Hilfe einer import-Anweisung eingebunden: \\
                  \texttt{import java.util.Date;} \\
                  \texttt{// \dots} \\
                  \texttt{Date d = new Date();}
            \item Bei den import-Anweisungen sollten kein Wildcards verwendet werden
            \item Am besten wildcard imports in der IDE abschalten
        \end{itemize}
    \end{block}
\end{frame}

\begin{frame}{Pakete}
    \begin{block}{Nützliches}
        \begin{itemize}
            \item Alle Klassen im Paket java.lang sind für die Sprache so essentiell, dass sie von jeder Klasse automatisch importiert werden
            \item Pakete können selber wieder beliebig viele Unterpakete haben
            \item Das Importieren von nicht benötigten Klassen kann den eigenen Namensraum stören (Namenskonflikte)
        \end{itemize}
    \end{block}
\end{frame}

