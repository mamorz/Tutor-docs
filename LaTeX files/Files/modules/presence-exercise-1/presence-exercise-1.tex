\section{Vorrechnen}

\begin{frame}{Präsenzübung: Vorechnnen}
    \begin{block}{Konzept 1}
        Zähle alle Schleifenvarianten auf und konvertiere folgende while-Schleife in eine for-Schleife \texttt{int i = 0; int sum = 0; while(i < 42) \{sum += i; i += 2;\}}. Versehe die drei wichtigen Komponenten mit einem Entwicklerkommentar.
    \end{block}
\end{frame}

\begin{frame}{Präsenzübung: Vorechnnen}
    \begin{block}{Konzept 2}
        Was ist an folgendem ungeschickt und wie geht es besser? \texttt{boolean value = getValue(); if(value == true) \{ return true; \} else \{ return false; \}}. Welches Ergebnis liefert \texttt{(!(true \& false) != true) || !((1 * 3) == 2)}?
    \end{block}
\end{frame}

\begin{frame}{Präsenzübung: Vorechnnen}
    \begin{block}{Konzept 3}
        Sei eine Methode mit dem Schlüsselwort \texttt{static} versehen. Wie kann diese Methode aufgerufen werden, vor bzw. nachdem das Schlüsselwort entfernt wurde? Wie wird ein neues Objekt erstellt? Schreibe einen kleinen Konstruktor für einen Studenten mit den Attributen Matrikelnummer, Wohnort, Vor- und Nachnamen.
    \end{block}
\end{frame}

\begin{frame}{Präsenzübung: Vorechnnen}
    \begin{block}{Zur Aufgabe 1}
        Aufgabe A: Wie sollte/hast du das Veröffentlichungsdatum modelliert? Füge dem Künstler ein Geburtstag hinzu.
    \end{block}
\end{frame}

\begin{frame}{Präsenzübung: Vorechnnen}
    \begin{block}{Zur Aufgabe 2}
        Aufgabe A: Könnte statt \texttt{public} auch \texttt{protected} für eine Klasse gewählt werden? Wie kann auf private Attribute zugegriffen werden und welche Probleme können dabei auftreten?
    \end{block}
\end{frame}

\begin{frame}{Präsenzübung: Vorechnnen}
    \begin{block}{Zur Aufgabe 3}
        Aufgabe B: Was ist der erste \& zweite Kommandozeilenparameter, woher kommen diese und wie werden sie aufgerufen?
    \end{block}
\end{frame}

\begin{frame}{Präsenzübung: Vorechnnen}
    \begin{block}{Zur Aufgabe 4}
        Aufgabe B: Wie werden führende Nullen ergänzt? Wie kann ich die Anzahl der Stellen um zwei erhöhen?
    \end{block}
\end{frame}

\begin{frame}{Präsenzübung: Vorechnnen}
    \begin{block}{Zur Aufgabe 5}
        Aufgabe C: Wie wurde die Matrix eingelesen und gespeichert? Was müsste geändert werden, damit die Matrix Gleitkommazahlen in doppelter Genauigkeit einliest bzw. enthält? Welche Probleme können in den Berechnungen auftreten?
    \end{block}
\end{frame}

\begin{frame}{Präsenzübung: Vorechnnen}
    \begin{block}{Zur Aufgabe 6}
        Aufgabe C: Gebe eine Fehlermeldung aus, wenn ein dritter Parameter übergeben wird. Füge einen weiteren Befehl print für übergebene Matrizen hinzu: \texttt{print 1,2,3;4,5,6;7,8,9}
    \end{block}
\end{frame}

\begin{frame}{Präsenzübung: Vorechnnen}
    \begin{block}{Konzept 4}
        Erkläre ob die folgende Operation komplett in short stattfindet: \texttt{short x = (short) 7 + 11;} und welches Ergebnis \texttt{int y = 'c' - 'a'} liefert. Schreibe ein Javadoc für die Matrix-Methode \texttt{public static Matrix multiply(scalar n, Matrix m) \{ \dots \}}.
    \end{block}
\end{frame}
