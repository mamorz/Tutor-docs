\section{Vorrechnen}

\begin{frame}{Präsenzübung: Vorechnnen}
    \begin{block}{Konzept 1}
        Schreibe eine Klasse „Temperatur“, die eine Gleitkommazahl in doppelter Genauigkeit enthält, die die Temperatur repräsentiert. Füge ein Attribut Bereich hinzu, mit einem Enum mit den Kategorien kalt, warm, heiß.
    \end{block}
\end{frame}

\begin{frame}{Präsenzübung: Vorechnnen}
    \begin{block}{Konzept 2}
        Schreibe zur vorherigen Klasse eine toString-Methode mit Javadoc, die das Attribut \& die Kategorie ausgibt.
    \end{block}
\end{frame}

\begin{frame}{Präsenzübung: Vorechnnen}
    \begin{block}{Konzept 3}
        Schreibe zur vorherigen Klasse eine Erwärmen-Methode, die die Temperatur um 10.0 °C erhöht, speichert und bei über 50.0 °C „Autsch!“ auf der Konsole ausgibt.
    \end{block}
\end{frame}

\begin{frame}{Präsenzübung: Vorechnnen}
    \begin{block}{Konzept 4}
        Schreibe zur vorherigen Klasse ein Konstruktor für die Gleitkommazahl – der Bereich kann nun ignoriert werden – und eine main-Methode, die die Klasse mit 35.3 instantiiert und anschließend die Erwärmen-Methode abruft. Schreibe innerhalb der Methode einen einzeiligen Entwicklerkommentar: „TODO Schnelles lokales testen. Nicht mit abgeben!“
    \end{block}
\end{frame}

\begin{frame}{Präsenzübung: Vorechnnen}
    \begin{block}{Konzept 5}
        Füge der main-Methode ein double-Array test mit drei beliebigen Zahlen hinzu, iteriere über diese Werte mit einer geeigneten Schleife und instantiiere jeweils die Temperatur-Klasse. Schreibe darüber ein mehrzeiligen Entwicklerkommentar: „Um ganz sicher zu gehen.“ „Der Übungsschein gehört mir!“
    \end{block}
\end{frame}

\begin{frame}{Präsenzübung: Vorechnnen}
    \begin{block}{Konzept 6}
        Ergänze zwei Klassenkonstanten mit den Zeichenketten „Weihnachten“ \& „Anmeldung“. Überprüfe in der main-Methode auf den ersten Kommandozeilenparameter mit einem switch-case und gebe „Ich wünsche frohe Weihnachten!“, „Anmeldung für den Übungsschein nicht vergessen!“ bzw. „Nichts.“ aus. Es wird garantiert ein Kommandozeilenparameter übergeben.
    \end{block}
\end{frame}

\begin{frame}{Präsenzübung: Vorechnnen}
    \begin{block}{Konzept 7}
        Schreibe eine öffentliche statische Methode, die drei Temperatur-Klassen als Parameter übergeben bekommt und den jeweils höchsten zurückgibt.
    \end{block}
\end{frame}

\begin{frame}{Präsenzübung: Vorechnnen}
    \begin{block}{Konzept 8}
        Die Wichtel waren unvorsichtig und haben zwei Gleitkommazahlen direkt miteinander verglichen. Erkläre was dabei ungeschickt ist, und wie dies umgangen werden kann. Außerdem möchten sie den folgenden Ausdruck verstehen: \texttt{ (!!(((1.0 > 1.2) \& false) || (\"Schnee\" == \"Schnee\"))) \&\& ((6 * 9) == 42)}
    \end{block}
\end{frame}

\begin{frame}{Präsenzübung: Vorechnnen}
    \begin{block}{Konzept 9}
        Zähle vier primitive Datentypen auf. Erkläre Schritt für Schritt, was hier geschieht: \\
        \texttt{int x = 1337; short y = 'D' - 'A'; int z = x + y; x = z++;}
    \end{block}
\end{frame}

\begin{frame}{Präsenzübung: Vorechnnen}
    \begin{block}{Konzept 10}
        Zähle die verschiedenen Sichtbarkeiten in Java auf, inklusive ihrem Verwendungszweck. Erkläre kurz den Unterschied zwischen einer „Shallow Copy“ \& einer „Deep Copy“.
    \end{block}
\end{frame}
