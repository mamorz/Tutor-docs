\section{Vorrechnen}

\begin{frame}{Präsenzübung: Vorechnnen}
    \begin{block}{Zur Aufgabe 1}
        Aufgabe A: Wie hast du die Gewinnüberprüfung gelöst? Wie könnte dies auf eine Spielbrettgröße von 4x4 geändert werden?
    \end{block}
\end{frame}

\begin{frame}{Präsenzübung: Vorechnnen}
    \begin{block}{Zur Aufgabe 2}
        Aufgabe A: Ergänze eine toString-Methode in der Spielbrett-Klasse und rufe diese in der main-Methode auf. Schreibe dazu einen Entwicklerkommentar.
    \end{block}
\end{frame}

\begin{frame}{Präsenzübung: Vorechnnen}
    \begin{block}{Zur Aufgabe 3}
        Aufgabe B: Erkläre dein Vorgehen in der removeAccount-Methode und füge der Bank ein Attribut Motto hinzu, dass genau einmal im Konstruktor gesetzt wird und danach nicht veränderlich ist.
    \end{block}
\end{frame}

\begin{frame}{Präsenzübung: Vorechnnen}
    \begin{block}{Zur Aufgabe 4}
        Aufgabe B: Ergänze eine Klassenkonstante Warnung mit dem Wert 20 und gebe eine Warnung auf der Konsole aus, wenn nach einer Transaktion die Kontohöhe unter diesem Wert liegt.
    \end{block}
\end{frame}

\begin{frame}{Präsenzübung: Vorechnnen}
    \begin{block}{Zur Aufgabe 5}
        Aufgabe C: Erkläre wie die übergebene Datei verarbeitet bzw. gespeichert wird. Ergänze ein weiteres Kommandozeilenargument Name, dass zusammen mit dem Ergebnis ausgegeben wird.
    \end{block}
\end{frame}

\begin{frame}{Präsenzübung: Vorechnnen}
    \begin{block}{Zur Aufgabe 6}
        Aufgabe C: Erkläre die Berechnung des Endergebnis. Zähle die vier Schleifenarten auf und wandle eine beliebige Schleife in deinem Programm in eine andere um.
    \end{block}
\end{frame}

\begin{frame}{Präsenzübung: Vorechnnen}
    \begin{block}{Konzept 1}
        Schreibe eine Methode inklusive Javadoc, die eine Zahl entgegen nimmt, diese um 2 erniedrigt und folgend ausgibt „Anmeldung des Übungsschein nicht vergessen! Verbleibende Tage: <Zahl>“ oder wenn die Zahl nach Erniedrigung unter oder auf 7 liegt denselben Text in reinen Großbuchstaben.
    \end{block}
\end{frame}

\begin{frame}{Präsenzübung: Vorechnnen}
    \begin{block}{Konzept 2}
        Zähle vier primitive Datentypen auf. Erkläre Schritt für Schritt, was hier geschieht: \\
        \texttt{int x = 42; short y = 'D' - 'B'; int z = x * y; x = ++z;}
    \end{block}
\end{frame}
