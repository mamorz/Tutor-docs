\section{Vorrechnen}

\begin{frame}{Präsenzübung: Vorechnnen}
    \begin{block}{Konzept 1}
        Schreibe eine Methode, die eine Zahl entgegen nimmt, diese um 2 erniedrigt und folgend ausgibt „Anmeldung des Übungsschein nicht vergessen! Verbleibende Tage: <Zahl>“ oder wenn die Zahl nach Erniedrigung unter oder auf 7 liegt denselben Text in reinen Großbuchstaben.
    \end{block}
\end{frame}

\begin{frame}{Präsenzübung: Vorechnnen}
    \begin{block}{Konzept 2}
        Zähle vier primitive Datentypen auf. Erkläre Schritt für Schritt, was hier geschieht: \\
        \texttt{int x = 42; short y = 'D' - 'B'; int z = x * y; x = ++z;} \\
        Schreibe zusätzlich ein Javadoc für eine main-Methode mit dem ersten Kommandozeilenparameter als Altersangabe.
    \end{block}
\end{frame}

\begin{frame}{Präsenzübung: Vorechnnen}
    \begin{block}{Konzept 3}
        Schreibe eine Schleife, die solange Nutzereingaben entgegennimmt bis 'quit' eingegeben wird und sonst ausgibt: „Der Rückmeldezeitraum hat begonnen. Zahle 151.20 € in die Kasse des KIT.“ In welcher Form sollte die Zahlungshöhe geschickterweise hinterlegt werden?
    \end{block}
\end{frame}

\begin{frame}{Präsenzübung: Vorechnnen}
    \begin{block}{Konzept 4}
        Schreibe eine Methode, die ein Array aus Gleitkommazahlen in doppelter Genauigkeit entgegennimmt und dieses umgekehrt wieder ausgibt. Ergänze einen beliebigen einzeiligen Entwicklerkommentar. Welcher grundsätzliche Verdacht besteht bei Rückgabe komplexer Datentypen, darunter auch Arrays?
    \end{block}
\end{frame}

\begin{frame}{Präsenzübung: Vorechnnen}
    \begin{block}{Konzept 5}
        Zähle die vier Sichtbarkeiten und kurz ihre Auswirkung auf. Was bedeuten die Schlüsselworte \texttt{final} \& \texttt{static}? Schreibe sinnvoll ein Attribut für ein Geburtsdatum einer Person, sowie den passenden Konstruktor.
    \end{block}
\end{frame}

\begin{frame}{Präsenzübung: Vorechnnen}
    \begin{block}{Konzept 6}
        Was sind Utility-Klassen? Was muss dort beachtet werden? Schreibe eine equals-Methode für einen Punkt eines zweidimensionalen Koordinatensystem. Ergänze einen mehrzeiligen Entwicklerkommentar wie viel Freude dir Programmieren doch bereitet.
    \end{block}
\end{frame}

\begin{frame}{Präsenzübung: Vorechnnen}
    \begin{block}{Wildcard}
        Wähle eine beliebige Aufgabe vom zweiten bis zum vierten Blatt und gehe auf deine Herangehensweise und den Aufbau ein. Rechne mit Gegen- und Zwischenfragen.
    \end{block}
\end{frame}
