\graphicspath{{../modules/programming-introduction/img/}}

\section{Programmieren}

\subsection{\frqq Schöner programmieren\flqq}

\begin{frame}{Programmieren - aber \frqq schön\flqq}
    \begin{block}{\frqq Schöner programmieren\flqq}
        \begin{itemize}
            \item Warum sollte man \frqq sauber\flqq programmieren?

                  \pause

            \item Was ist eigentlich \frqq sauber\flqq programmieren?

                  \pause

                  \begin{itemize}
                      \item Namen / Bezeichner sind aussagekräftig und in Englisch
                      \item Kommentare existieren und sind sinnvoll
                      \item Einheitliche Sprache in Kommentaren
                      \item Sourcecode enthält sinnvolle Einrückungen
                  \end{itemize}
        \end{itemize}
    \end{block}

    \pause

    \begin{exampleblock}{So?}
        \lstinputlisting[]{../modules/programming-introduction/src/demo-bad.java}
    \end{exampleblock}
\end{frame}

\begin{frame}{Programmieren - aber \frqq schön\flqq II}
    \begin{exampleblock}{Oder doch lieber so?}
        \lstinputlisting[]{../modules/programming-introduction/src/demo-nice.java}
    \end{exampleblock}
\end{frame}

\begin{frame}{Programmieren - aber \frqq schön\flqq III}
    \begin{block}{Hilfe}
        \begin{itemize}
            \item Checkstyle!
            \item Automatische Überprüfung, abgaberelevant!
            \item Formatter helfen euch
            \item Tipp: Später noch verwenden, bspw. Google Code Style
        \end{itemize}
    \end{block}
\end{frame}

\subsection{Klassen und Objekte}

\begin{frame}
    \begin{center}
        \begin{huge}
            Klassen und Objekte
        \end{huge}
    \end{center}
\end{frame}

\begin{frame}{Wozu Klassen und Objekte?}
    \begin{itemize}

        \pause

        \item Klassen und Objekte sollen uns helfen, von der Realität zu abstrahieren. Diese also möglichst einfach, aber auch weitestgehend vollständig (soweit machbar / sinnvoll) abzubilden. \\

              \begin{center}
                  \includegraphics[scale=0.25]{class-object}
              \end{center}

    \end{itemize}
\end{frame}

\begin{frame}{Unterschied Klasse / Objekt}
    \begin{block}{Klasse}
        Ideen?
    \end{block}
\end{frame}

\begin{frame}{Unterschied Klasse / Objekt}
    \begin{block}{Klasse}
        \parpic[r]{\includegraphics[scale=0.25]{class}}

        \begin{itemize}
            \item Eine Klasse ist der \frqq Bauplan\flqq\ für gleichartige Objekte und legt für diese fest:

                  \begin{itemize}
                      \item \textit{Attribute}
                      \item \textit{Methoden}
                  \end{itemize}

            \item Schema:
                  \lstinputlisting[]{../modules/programming-introduction/src/class-scheme.java}
            \item Dateiname = Klassenname (Buch.java für Klasse Buch)
        \end{itemize}

    \end{block}
\end{frame}

\begin{frame}{Unterschied Klasse / Objekt}
    \begin{block}{Objekt}
        Ihr seid wieder dran!
    \end{block}
\end{frame}

\begin{frame}{Unterschied Klasse / Objekt II}
    \begin{block}{Objekt}
        \parpic[r]{\includegraphics[scale=0.25]{object}}

        \begin{itemize}
            \item Mit der Klasse als \frqq Bauplan\flqq\ können nun viele Objekte \frqq gebaut\flqq\ werden.
            \item \textit{Attribute} bestimmen den \textit{Zustand} des Objekts,
            \item \textit{Methoden} das \textit{Verhalten}
            \item Besitzen eine Identität, die verglichen \textit{(==)} werden kann

                  \pause

            \item Was macht ein Vergleich mit \textit{(==)} bei primitiven Datentypen?
        \end{itemize}

    \end{block}
\end{frame}

\subsection{Beispielklasse}

\begin{frame}{Klasse - Ein Beispiel}
    \begin{exampleblock}{Beispiel: Student}
        \begin{itemize}
            \item Welche Informationen benötigen wir für einen Studenten?

                  \pause

            \item z.B.: Matrikelnummer, Name, Vorname, Strasse, Hausnummer, Postleitzahl, Ort, Telefon, Email, Studiengang, Fachsemester, \dots

                  \pause

            \item Das könnte dann wie folgt aussehen:\\
                  \lstinputlisting[basicstyle=\tiny,language=java]{../modules/programming-introduction/src/student.java}
        \end{itemize}
    \end{exampleblock}
\end{frame}
