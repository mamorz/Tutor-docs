\section{Wiederholung}

\begin{frame}{Wiederholung}
    \begin{block}{Bisherige Inhalte}
        \begin{itemize}
            \item Klassen
            \item Objekte
            \item Attribute
            \item Variablen
            \item Datentypen
            \item enum
        \end{itemize}
    \end{block}

    \begin{exampleblock}{Wer weiss es?}
        Was war das alles gleich nochmal? (eigene Worte, Beispiel)
    \end{exampleblock}
\end{frame}

\begin{frame}{Wiederholung - Objekte}
    \begin{block}{Was sind die Eigenschaften eines Objektes?}
        \begin{itemize}
            \item Identität
            \item Zustand
            \item Verhalten
        \end{itemize}
    \end{block}
\end{frame}

\begin{frame}{Wiederholung - Objekte}
    \begin{exampleblock}{Ja, Nein oder Jein?}
        \texttt{==} kann zum Vergleichen von Objekten benutzt werden
    \end{exampleblock}

    \pause

    \begin{alertblock}{Jein}
        \texttt{==} vergleicht die Objekt-\textbf{Identität}.
        Dies ist meistens nicht gewünscht.
    \end{alertblock}
\end{frame}

\begin{frame}{Wiederholung - Datentypen}
    \begin{exampleblock}{Wahr oder Falsch?}
        Bei Operationen auf verschieden großen Datentypen wandelt Java den kleineren in den größeren Typ um.
    \end{exampleblock}

    \pause

    \begin{block}{Wahr}
        Java wandelt kleinere Datentypen in den Typ des größeren um, falls beide in einer Operation benutzt werden: \\
        \texttt{4 + (long) 5 = 9} wandelt \texttt{4} in einen \texttt{long} und gibt einen \texttt{long} zurück
    \end{block}
\end{frame}

\begin{frame}[fragile]{Wiederholung - Datentypen}
    \begin{exampleblock}{Wahr oder Falsch?}
        Variablen von jedem Datentyp können \texttt{null} als Wert annehmen.
    \end{exampleblock}

    \pause

    \begin{alertblock}{Falsch}
        Nur Variablen von Referenzdatentypen (Klassen) können \texttt{null} als Wert annehmen. \\
        \begin{lstlisting}
int    a = null; // Fehler, int   ist keine Klasse
@@cString@@ b = null; // Okay  , String ist eine Klasse
        \end{lstlisting}
    \end{alertblock}
\end{frame}

\begin{frame}[fragile]{Wiederholung - Datentypen}
    \begin{exampleblock}{Wahr oder Falsch?}
        Enums können keine Methoden enthalten.
    \end{exampleblock}

    \pause

    \begin{alertblock}{Falsch}
        Enums sind auch Klassen und können Variablen und Methoden enthalten.
    \end{alertblock}
\end{frame}

\begin{frame}{Wiederholung - Operatoren}
    \begin{exampleblock}{Was ist Kurzschlussauswertung?}

        \pause

        Wenn das Ergebnis eines Vergleiches bereits nach dem Auswerten der linken Seite feststeht, wird die rechte Seite nicht ausgewertet
    \end{exampleblock}
\end{frame}

\begin{frame}{Wiederholung - Operatoren}
    \begin{exampleblock}{Was ist der Unterschied zwischen \texttt{\&} und \texttt{\&\&}?}

        \pause

        \texttt{\&\&} nutzt \textbf{Kurzchlussauswertung},\& nicht
        \\ \ \\
        \texttt{||} analog\\
        Die einzelnen Operatoren sind die bitweisen
    \end{exampleblock}
\end{frame}

\begin{frame}[fragile]{Wiederholung - Operatoren}
    \begin{exampleblock}{Was ist der Unterschied zwischen \texttt{i++} und \texttt{++i}?}

        \pause

        \begin{itemize}
            \item \texttt{i++} gibt den momentanen Wert von \texttt{i} zurück und erhöht \texttt{i} danach
            \item \texttt{++i} gibt den bereits erhöhten Wert von \texttt{i} zurück
            \item \begin{lstlisting}
int x = 5;
int y = x++; // x = 6, y = 5

int x = 5;
int y = ++x; // x = 6, y = 6
            \end{lstlisting}
        \end{itemize}
    \end{exampleblock}
\end{frame}
