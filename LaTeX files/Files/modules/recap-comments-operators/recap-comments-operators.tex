\graphicspath{{../modules/recap-comments-operators/img/}}

\section{Wiederholung}

\begin{frame}{Was bisher geschah \dots}
    \begin{block}{Bisherige Inhalte}
        \begin{itemize}
            \begin{multicols}{2}
                \item Klassen
                \item Objekte
                \item Attribute
                \item Variablen
                \item Datentypen
                \item enum
                \item Methoden
                \item Konstruktoren
                \item main-Methode
                \item Kontrollstrukturen
                \begin{itemize}
                    \item Bedingte Ausführung
                    \item Schleifen
                \end{itemize}
            \end{multicols}
        \end{itemize}
    \end{block}

    \begin{exampleblock}{Wer weiss es?}
        In eigenen Worten, mit Beispiel?
    \end{exampleblock}
\end{frame}

\begin{frame}{Wiederholung}
    \begin{exampleblock}{\texttt{static}}
        \begin{itemize}
            \item Was war nochmal gleich der Unterschied zwischen mit/ohne \texttt{static}?

                  \pause

            \item Wann \texttt{static} und wann nicht?

                  \pause

            \item Wie ist das bei Methoden?

                  \pause

                  \texttt{new Math().abs(-20)} oder \texttt{Math.abs(-20)}?
        \end{itemize}
    \end{exampleblock}

    \pause

    \begin{exampleblock}{Kommentare?}
        \pause
        \begin{itemize}
            \item Zeilenkommentar: \texttt{// Ich bin ein Kommentar}
            \item Blockkommentar: \texttt{/* Mehrere Zeilen möglich! */}
            \item Javadoc: \texttt{/** Beschreibung einer Klasse, Methode, \dots */}
        \end{itemize}
    \end{exampleblock}
\end{frame}

\begin{frame}{Wiederholung}
    \begin{block}{Kommentare - Hinweise}
        \begin{itemize}
            \item Knapp aber präzise
            \item Keine offensichtlichen Banalitäten kommentieren, die direkt ersichtlich sind
            \item Zum Verständnis des Programms beitragen -- nicht dieses unleserlich machen
        \end{itemize}
    \end{block}
\end{frame}

\begin{frame}{Wiederholung}
    \begin{block}{Was ist das jeweilige Ergebnis?}
        \lstinputlisting[basicstyle=\small]{../modules/recap-comments-operators/src/expression-1a.java}
    \end{block}

    \pause

    \begin{alertblock}{Gefällts?}
        Schön, so ganz ohne Leerzeichen, nicht?\\
        Würdet Ihr solchen Code \sout{korrigieren} \textit{lesen} wollen?
        \\ \ \\
        \textit{(Just because you can doesn't mean you \textbf{should}.)}
    \end{alertblock}
\end{frame}

\begin{frame}{Wiederholung}
    \begin{block}{Was ist das jeweilige Ergebnis? \textit{(mit Hilfe)}}
        \lstinputlisting[basicstyle=\small]{../modules/recap-comments-operators/src/expression-1b.java}
    \end{block}

    \begin{exampleblock}{Ergebnis \dots}
        \pause

        i == -1215761 (Überlauf!)\\
        b == true
    \end{exampleblock}
\end{frame}

\begin{frame}{Wiederholung}
    \begin{block}{Was ist das Ergebnis für \textit{j1, c1, j2, c2}?}
        \lstinputlisting[basicstyle=\small]{../modules/recap-comments-operators/src/expression-2.java}
    \end{block}

    \begin{exampleblock}{Ergebnis \dots}

        \pause

        \begin{minipage}{0.5\textwidth}
            j1 = 0 \\
            c1 = true \\
            j2 = 4 \\
            c2 = false
        \end{minipage}% necessary percent sign for side-by-side content
        \begin{minipage}{0.5\textwidth}
            WARUM? \pause (Stichwort: Kurzauswertung)
        \end{minipage}
    \end{exampleblock}
\end{frame}

\begin{frame}{Aber warum? Was ist mit Präzedenz??}
    \begin{block}{Auswertung von Unterausdrücken}
        Die Auswertung von Unterausdrücken ist unabhängig von der Präzedenz der
        einzelnen Operatoren und wird meistens von links nach rechts evaluiert.
        Dabei ist garantiert, dass beide Argumente von Operatoren vor diesen
        ausgewertet werden (außer bei Kurzschlussauswertung).

        \vspace{1em}

        \begin{minipage}{0.4\textwidth}
            \includegraphics[scale=0.3]{example-from-before.png}
        \end{minipage}% necessary percent sign for side-by-side content
        \begin{minipage}{0.6\textwidth}
            Durch die von links nach rechts ablaufende Auswertung wird die Rechte Seite des \texttt{||} -- Operators nicht ausgewertet.
        \end{minipage}
    \end{block}
\end{frame}

\begin{frame}{Wiederholung}
    \begin{block}{Was ist das Ergebnis für \textit{s}?}
        \lstinputlisting[basicstyle=\small]{../modules/recap-comments-operators/src/expression-3a.java}
    \end{block}

    \begin{exampleblock}{Ergebnis \dots}

        \pause

        'Test: 535'
        \\ \ \\
        WARUM? \\

        \pause

        Präzedenz führt zu folgender Auswertung:

        \lstinputlisting[basicstyle=\small]{../modules/recap-comments-operators/src/expression-3b.java}

        Was müssen wir tun, um 'Test: 40' zu erhalten?
    \end{exampleblock}
\end{frame}
