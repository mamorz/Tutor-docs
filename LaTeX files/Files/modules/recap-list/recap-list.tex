\graphicspath{{../modules/recap-list/img/}}
\section{Was bisher geschah \dots}
\begin{frame}{Was bisher geschah \dots}
    \begin{block}{Bisherige Inhalte}
        \begin{itemize}
            \begin{multicols}{2}
                \item Klassen, Datentypen \& Objekte
                \item Variablen \& Attribute
                \item Methoden \& Konstruktoren
                \item Kontrollstrukturen
                \item Präzedenz
                \item Arrays \& Strings
                \item Kommentieren (incl. JavaDoc)
                \item Checkstyle (Whitespace)
                \item Referenzen
                \item Shallow Copy / Deep Copy
            \end{multicols}
        \end{itemize}
    \end{block}
\end{frame}

\section{Wdh. Listen}

\begin{frame}{Wdh. Listen}
    \setbeamercovered{invisible}
    \begin{block}{Was war das nochmal?}
        \begin{itemize}
            \item Abstrakte Datenstruktur zur speicherung von Daten
            \pause
            \item Können mehrere Elemente spichern
            \pause
            \item Dynamische Länge!
            \pause
            \item Im Prinzip intuitiv wie eine \glqq Einkaufsliste''
        \end{itemize}
    \end{block}
\end{frame}

\begin{frame}{Wdh. Listen}
    \setbeamercovered{invisible}
    \begin{block}{Was für Listen gibt es und was ist das Besondere?}
        \begin{itemize}
            \item Single linked Lists
            \aitem Jedes Element hält nur eine Referenz auf das nächste Element
            \pause
            \item Double linked Lists
            \aitem Jedes Element hält eine Referenz auf das nächste und das letzte Element
            \pause
            \item Array Lists
            \aitem Funktionieren intern mit Arrays und kopieren bei Bedarf Daten rum. 
            Daher sind sie (oft) schneller.
        \end{itemize}
    \end{block}
\end{frame}

\begin{frame}{Wdh. Listen}
    \setbeamercovered{invisible}
    \begin{block}{Welche Operationen auf Listen gibt es?}
        \begin{itemize}
            \item \lstinline{insert()}
            \item \lstinline{removeAfter(int index)}
            \item \lstinline{contains(E element)}
            \item \lstinline{get(int index)}
            \item \dots
        \end{itemize}
    \end{block}
\end{frame}

\begin{frame}{Wdh. Listen}
    \setbeamercovered{invisible}
    \begin{block}{Welche Operationen könnten potentiell Probleme bereiten?}
        \begin{itemize}
            \item \lstinline{insertBefore()} 
            \aitem Blöd wenn man nur einfach verkettete Liste hat 
            \item \lstinline{getLast(): E}
            \aitem Teuer auf single linked lists, da man alle Elemente durchlaufen muss
            \item \lstinline{contains(E element)}
            \aitem Man muss potentiell alle Elemente durchgehen
        \end{itemize}
    \end{block}
\end{frame}

\subsection{Single linked lists}

\begin{frame}{Single linked lists}
    \begin{block}{Was war das nochmal?}
        \begin{itemize}
            \item ListElements, halten jeweils ein bestimmtes Datum und eine Referenz auf das nächste Objekt
            \item Listenklasse hält Referenz auf erstes ListElement und weitere Daten, wie zum Beispiel Länge
        \end{itemize}
    \end{block}
    \begin{center}
        \includegraphics[scale=0.5]{linkedList}        
    \end{center}
\end{frame}

\begin{frame}{Double linked lists}
    \begin{block}{Was war das nochmal?}
        \begin{itemize}
            \item ListElements, halten jeweils ein bestimmtes Datum und, Referenz auf das nächste Objekt \textbf{und} eine Referenz auf das letzte Objekt
            \item Listenklasse hält Referenz auf erstes ListElement, letztes ListElement und weitere Daten, wie zum Beispiel Länge
        \end{itemize}
    \end{block}
    \begin{center}
        \includegraphics[scale=0.35]{doubleLinkedList}        
    \end{center}
\end{frame}

\begin{frame}{Ein paar Operationen auf SingleLinkedLists}
    \setbeamercovered{invisible}
    \begin{block}{\lstinline{addFirst(E elem)}}
        \begin{itemize}

            \pause

            \item Neues Element erzeugen
            \item Randfälle überprüfen (ist die Liste leer?)
            \item Den next Zeiger des neuen Elementes auf den first Zeiger der Liste setzen
            \item Den first Zeiger der Listenklasse updaten
            \item Zähler für die Länge inkrementieren
        \end{itemize}
    \end{block}
\end{frame}

\begin{frame}{addFirst - Neues Element erzeugen}
    \begin{center}
        \includegraphics[scale=0.5]{addFirst/linkedList_new_elem.png}
    \end{center}
\end{frame}

\begin{frame}{addFirst - Randfälle prüfen}
    \begin{center}
        \includegraphics[scale=0.5]{addFirst/linkedList_new_elem.png}\\
        \large $\rightarrow$ Liste ist nicht leer.
    \end{center}
\end{frame}

\begin{frame}{addFirst - Zeiger des neuen Elementes updaten}
    \begin{center}
        \includegraphics[scale=0.5]{addFirst/linkedList_pointer_updated.png}
    \end{center}
\end{frame}

\begin{frame}{addFirst - first Zeiger der Listenklasse updaten}
    \begin{center}
        \includegraphics[scale=0.5]{addFirst/linkedList_inserted.png}
    \end{center}
\end{frame}

\begin{frame}{addFirst - Zähler inkrementieren}
    \begin{center}
        \includegraphics[scale=0.5]{addFirst/linkedList_complete.png}
    \end{center}
\end{frame}


\begin{frame}{Ein paar Operationen auf SingleLinkedLists}
    \setbeamercovered{invisible}
    \begin{block}{\lstinline{delete(int index)}}
        \begin{itemize}

            \pause

            \item Element am Index vorher finden (falls es existiert)
            \item next Zeiger vom Element umbiegen (dabei Randfälle beachten)
            \item Da keine Referenzen auf das Objekt mehr existieren, wird es vom Java garbage colector gelöscht
        \end{itemize}
    \end{block}
\end{frame}

\begin{frame}{\textit{delete(2)} - Element am Index vorher finden}
    \begin{center}
        \includegraphics[scale=0.4]{delete/linkedListElementLocated.png}
    \end{center}
\end{frame}

\begin{frame}{\textit{delete(2)} - Next Zeiger umbiegen}
    \begin{center}
        \includegraphics[scale=0.4]{delete/linkedListNextChanged.png}
    \end{center}
\end{frame}

\begin{frame}{\textit{delete(2)} - Garbage collector entsorgt Objekt}
    \begin{center}
        \includegraphics[scale=0.4]{delete/linkedListFinished.png}
    \end{center}
\end{frame}