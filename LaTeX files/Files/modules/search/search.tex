\section{Suchen}

\begin{frame}{Suchen}
    \begin{block}{Allgemein}
        \begin{itemize}
            \item Herausfinden, ob und wo ein Element in einem Array vorkommt
            \item Verschiedene Ansätze \dots
        \end{itemize}
    \end{block}

    \begin{block}{Lineare Suche}
        \begin{itemize}
            \item Keine Vorraussetzungen
            \item Array wird durchlaufen, bis das Element gefunden (oder das Ende erreicht) wurde
        \end{itemize}
    \end{block}

    \begin{block}{Binäre Suche}
        \begin{itemize}
            \item Vorraussetzung: Array muss sortiert sein
            \item Vorteil: Im Normalfall schneller als lineare Suche
            \item Wähle die Mitte, Gefunden? Größer? Kleiner? Entsprechend von vorne beginnen für neuen Bereich
        \end{itemize}
    \end{block}
\end{frame}
