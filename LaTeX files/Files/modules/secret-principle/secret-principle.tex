\section{Geheimnisprinzip}

\begin{frame}{Geheimnisprinzip}
    \begin{block}{Allgemein}
        \begin{itemize}
            \item Programmierschnittstelle (API): Programmteil, welcher von einem Softwaresystem anderen Programmen zur Anbindung an das System zur Verfügung gestellt wird
            \item Programmbibliothek (Library): Sammlung von Unterprogrammen und Routinen, die Lösungswege für thematisch zusammengehörende Problemstellungen anbieten
            \item Zur Bereitstellung solcher Schnittstellen gehört eine detaillierte Dokumentation aller Funktionen samt ihren Parametern
        \end{itemize}
    \end{block}
\end{frame}

\begin{frame}{Geheimnisprinzip}
    \begin{block}{Datenkapselung}
        \begin{itemize}
            \item Attribute werden vor dem Zugriff von außen vorborgen: Der direkte Zugriff auf die interne Datenstruktur wird unterbunden und erfolgt stattdessen über definierte Methoden
            \item Abschotten der internen Implementierung vor direktem externen Zugriff: Dieser darf nur über eine explizit definierte Schnittstelle erfolgen, um ihn unabhängig von den Implementierungsdetails zu machen
            \item Vorteile:

                  \begin{itemize}
                      \item Die Implementierung von Klassen kann geändert werden, ohne die Zusammenarbeit mit anderen Klassen zu beeinträchtigen
                      \item Objekte können den internen Zustand anderer Objekte nicht in unerwarteter Weise lesen oder ändern
                  \end{itemize}
        \end{itemize}
    \end{block}
\end{frame}

\begin{frame}{Geheimnisprinzip}
    \begin{block}{Sichtbarkeit}
        \begin{itemize}
            \item Mit Zugriffsmodifikatoren lassen sich die Sichtbarkeiten von Programmteilen regeln:
                  \begin{itemize}
                      \item public: Element ist für alle Klassen sichtbar
                      \item private: Element ist nur innerhalb seiner Klasse sichtbar
                      \item protected: Element ist nur innerhalb seiner Klasse, deren Subklassen und allen Klassen im selben Paket sichtbar
                      \item default: Kein Modifikator bedeutet, dass das Element nur innerhalb seiner Klasse und der Klassen im selben Paket sichtbar ist
                  \end{itemize}
        \end{itemize}
    \end{block}
\end{frame}

\begin{frame}{Geheimnisprinzip}
    \begin{block}{Ab sofort}
        \begin{itemize}
            \item Klassen sind in der Regel immer public
            \item Sämtliche Attribute einer Klasse sollten private sein!
            \item Bei Konstanten (static final) kann public sinnvoll sein
            \item Methoden sind in der Regel immer public
            \item Wenn Sie nur als lokale Hilfsmethoden gedacht sind, sollten sie private sein
        \end{itemize}
    \end{block}
\end{frame}

\begin{frame}{Geheimnisprinzip}
    \begin{block}{Sichtbarkeit \& Methoden}
        \begin{itemize}
            \item Schnittstellen-Methoden (public)

                  \begin{itemize}
                      \item Schützen das Klassengeheimnis
                      \item Bieten Abstraktion über Implementierungsdetails
                      \item Haben eine Aufgabe, die mit dem Namen zusammenhängt
                  \end{itemize}

            \item Hilfsmethoden (private)

                  \begin{itemize}
                      \item Sind funktionale und oder logische Einheiten in sich
                      \item Vermeiden Code-Redundanz
                  \end{itemize}
        \end{itemize}
    \end{block}
\end{frame}

\begin{frame}{Geheimnisprinzip}
    \begin{block}{Sichtbarkeit \& Pakete}
        \begin{itemize}
            \item Klassen kennen normalerweise nur die Klassen im eigenen Paket

            \item Falls eine Klasse ohne Paket-Angabe implementiert wird, befindet sie sich standardmäßig im unbenannten Paket

                  \begin{itemize}
                      \item Eine im Paket befindliche Klasse kann jede andere sichtbare Klasse aus anderen Paketen importieren, aber keine Klassen aus dem unbenannten Paket
                  \end{itemize}

            \item Kein Zugriffsmodifikatoren bedeutet, dass ein Element nur innerhalb seiner Klasse und der Klassen im selben Paket sichtbar ist.

                  \begin{itemize}
                      \item Verhält sich innerhalb eines Paketes wie: public
                      \item Verhält sich außerhalb eines Paketes wie: private
                  \end{itemize}
        \end{itemize}
    \end{block}
\end{frame}

\begin{frame}{Geheimnisprinzip}
    \begin{exampleblock}{Beispiel}
        \lstinputlisting[basicstyle=\small]{../modules/secret-principle/src/visibility-1.java}
    \end{exampleblock}
\end{frame}

\begin{frame}{Geheimnisprinzip}
    \begin{exampleblock}{Beispiel}
        \lstinputlisting[basicstyle=\small]{../modules/secret-principle/src/visibility-2.java}
    \end{exampleblock}
\end{frame}
