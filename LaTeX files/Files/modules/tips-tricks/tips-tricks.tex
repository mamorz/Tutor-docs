\graphicspath{{../modules/tips-tricks/img/}}

\section{Tipps \& Tricks}

\begin{frame}{Tipps \& Tricks}
    \begin{block}{Erst denken - dann programmieren!}
        \begin{itemize}
            \item Entwickle einen Plan vor der Implementierung
            \item „Divide and conquer“
            \item Stift und Papier verwenden (Aufbaudiagramme \& Pseudocode)
            \item Finden von Schnittstellen (Sichtbarkeit)
            \item Finden geeigneter Datenstrukturen
            \item Fragen entwickeln ($\rightarrow$ Forum!)
        \end{itemize}
    \end{block}
\end{frame}

\begin{frame}{Tipps \& Tricks}
    \begin{block}{UNIX-Philosophie}
        \begin{itemize}
            \item Modularität
            \item Klarheit
            \item Einfachheit
            \item Intuitiv
            \item Ebenso ansehenswert: SOLID-Prinzipien im \href{https://ilias.studium.kit.edu/goto.php?target=wiki_wpage_25914_996911&client_id=produktiv}{ILIAS-Wiki}
        \end{itemize}
    \end{block}
\end{frame}

\begin{frame}{Tipps \& Tricks}
    \begin{block}{Paketstruktur}
        \begin{itemize}
            \item Verwende Pakete zur Strukturierung
            \item Bsp: \texttt{edu.kit.informatik.connectfour.data.board}
        \end{itemize}
    \end{block}
\end{frame}

\begin{frame}{Tipps \& Tricks}
    \begin{block}{„Fail fast“}
        \begin{itemize}
            \item Mit Fehlern rechnen
            \item Fehler früh erkennen
            \item Konsistenten Zustand einer Klasse behalten
            \item Exceptions verwenden
        \end{itemize}
    \end{block}
\end{frame}

\begin{frame}{Tipps \& Tricks}
    \begin{exampleblock}{„Fail fast“ - Nein \dots}
        \lstinputlisting[basicstyle=\small]{../modules/tips-tricks/src/fail-fast-0.java}
    \end{exampleblock}
\end{frame}

\begin{frame}{Tipps \& Tricks}
    \begin{exampleblock}{„Fail fast“ - Nein \dots}
        \lstinputlisting[basicstyle=\small]{../modules/tips-tricks/src/fail-fast-1.java}
    \end{exampleblock}
\end{frame}

\begin{frame}{Tipps \& Tricks}
    \begin{exampleblock}{„Fail fast“ - JA!}
        \lstinputlisting[basicstyle=\small]{../modules/tips-tricks/src/fail-fast-2.java}
    \end{exampleblock}
\end{frame}

\begin{frame}{Tipps \& Tricks}
    \begin{block}{Programmieren gegen Schnittstellen}
        \begin{itemize}
            \item Wer erinnert sich? (Letztes Tutorium)

                  \pause

            \item \texttt{List<String> someList = Arraylist<>();}
        \end{itemize}
    \end{block}
\end{frame}

\begin{frame}{Tipps \& Tricks}
    \begin{block}{Verhalten nach Zustand}
        \begin{itemize}
            \item Objekte sollten sich ihrem Zustand entsprechend verhalten und nicht nach ihrem Zustand verwendet werden
            \item Richtig: Starte Auto ohne angeschnallt zu sein $\Rightarrow$ Auto erkennt und startet nicht
            \item Falsch: Überprüfe Gurt $\Rightarrow$ Starte Auto, falls angeschnallt
        \end{itemize}
    \end{block}
\end{frame}

\begin{frame}{Tipps \& Tricks}
    \begin{exampleblock}{Verhalten nach Zustand}
        \lstinputlisting[basicstyle=\small]{../modules/tips-tricks/src/behaviour.java}
    \end{exampleblock}
\end{frame}

\begin{frame}{Tipps \& Tricks}
    \begin{block}{Datenkapselung}
        \begin{itemize}
            \item Attribute immer \texttt{private}
            \item Sinnvolle Getter/Setter
            \item Überprüfen der übergebenen Werte
            \item Nur \texttt{public} wenn notwendig
            \item Nie Implementierungsdetails preisgeben
            \item Perspektive eines Angreifers einnehmen
        \end{itemize}
    \end{block}
\end{frame}

\begin{frame}{Tipps \& Tricks}
    \begin{block}{Datenstruktur vs. abstrakter Datentyp}
        \begin{itemize}
            \item Einen Datenstruktur ist eine konkrete Anordnung/Organisation/Speicherung von Daten im Rechner
            \item Abstrakte Datentypen hingegen verraten nichts über eine mögliche Implementierung
        \end{itemize}
    \end{block}
\end{frame}

\begin{frame}{Tipps \& Tricks}
    \begin{exampleblock}{Datenstruktur vs. abstrakter Datentyp?}
        \begin{itemize}
            \item Queue \only<2>{$\rightarrow$ abstrakter Datentyp}
            \item Stack \only<2>{$\rightarrow$ abstrakter Datentyp}
            \item Heap \only<2>{$\rightarrow$ Datenstruktur}
            \item Map \only<2>{$\rightarrow$ abstrakter Datentyp}
            \item PriorityQueue \only<2>{$\rightarrow$ abstrakter Datentyp}
            \item HashMap \only<2>{$\rightarrow$ Datenstruktur}
            \item Array \only<2>{$\rightarrow$ Datenstruktur}
            \item Graph \only<2>{$\rightarrow$ abstrakter Datentyp}
        \end{itemize}
    \end{exampleblock}
\end{frame}

\begin{frame}{Tipps \& Tricks}
    \begin{block}{Bezeichner}
        \begin{itemize}
            \item Aussagekräftige, nicht deutsche Bezeichner verwenden!
        \end{itemize}
    \end{block}
\end{frame}

\begin{frame}{Tipps \& Tricks}
    \begin{block}{Ausnahmesituationen}
        \begin{itemize}
            \item Exceptions sind Ausnahmesituationen, nicht Kontrollfluss!
            \item Eigene Exception schreiben
        \end{itemize}
    \end{block}
\end{frame}

\begin{frame}{Tipps \& Tricks}
    \begin{exampleblock}{InputException - Teil 1}
        \lstinputlisting[basicstyle=\tiny]{../modules/tips-tricks/src/input-exception-1.java}
    \end{exampleblock}
\end{frame}

\begin{frame}{Tipps \& Tricks}
    \begin{exampleblock}{InputException - Teil 2}
        \lstinputlisting[basicstyle=\tiny]{../modules/tips-tricks/src/input-exception-2.java}
    \end{exampleblock}
\end{frame}

\begin{frame}{Tipps \& Tricks}
    \begin{block}{Dokumentation}
        \begin{itemize}
            \item Quelltext dokumentieren und kommentieren!

                  \pause

            \item Kommentare beschreiben Logik, nicht Java-Syntax
            \item JavaDoc \dots
        \end{itemize}
    \end{block}
\end{frame}

\begin{frame}{Tipps \& Tricks}
    \begin{block}{Vermeide Tour-de-France-Quelltext}
        \begin{itemize}
            \item Lange, unübersichtlichen Quelltext in sinnvolle Methoden aufteilen
            \item Maximal 20 bis 30 Zeilen in einer Methode als Orientierung
            \item Tiefe Verschachtlungen vermeiden \href{https://blog.codinghorror.com/flattening-arrow-code/}{Click.}
        \end{itemize}
    \end{block}
\end{frame}

\begin{frame}{Tipps \& Tricks}
    \begin{block}{Enums?}
        \begin{itemize}
            \item Verwende Enums.
            \item Die können auch eigene Attribute, Konstruktoren, und Methoen haben
        \end{itemize}
    \end{block}
\end{frame}

\begin{frame}{Tipps \& Tricks}
    \begin{block}{Debuggen}
        \begin{itemize}
            \item Breakpoints (Haltepunkte) erleichtern das Finden von Fehlerursachen im Quelltext
            \item Idee: Programm in „Zeitlupe“ laufen lassen
            \item Vorgehensweise:

                  \begin{itemize}
                      \item Setze Breakpoints an „kritischen“ Stellen im Quelltext
                      \item Führe das Programm im Debug-Modus aus
                      \item Das Programm wird an den markierten Stellen pausieren
                      \item Der Programmierer hat die Chance, den Zustand des Programms zu untersuchen und nach Anomalien zu suchen (Inhalt von Variablen)
                      \item Beginne von Vorne
                  \end{itemize}

            \item Alle gängigen IDEs (verwendet eine!) unterstützen das Setzen von Breakpoints
        \end{itemize}
    \end{block}
\end{frame}

\begin{frame}{Tipps \& Tricks}
    \begin{block}{Versionskontrolle}
        \begin{itemize}
            \item Horrorszenario: Zwei Tage vor der Abgabe stirbt die Festplatte, kein Backup
            \item Präventive Maßnahme: (Cloud-)Backup \& Versionskontrolle
            \item Empfehlung: \href{https://git-scm.com/}{Git}
            \item \href{https://git.scc.kit.edu/}{GitLab-Instanz des SCC}
            \item Vielzahl von Cheatsheets \& Integration in Entwicklungsumgebung
        \end{itemize}
    \end{block}
\end{frame}

\begin{frame}{Tipps \& Tricks}
    \begin{block}{Checkstyle?}
        \begin{itemize}
            \item Benutzen (lol)
            \item In der IDE installieren und nicht einfach in Artemis hochladen bis die Abgabe klappt
        \end{itemize}
    \end{block}
\end{frame}

\begin{frame}{Tipps \& Tricks}
    \begin{block}{Testen?}
        \begin{itemize}
            \item TESTEN!
            
            \pause

            \item JUnit ist toll
            \item Früh testen um nicht am Ende alle Fehler gleichzeitig fixen zu müssen
            \item Wenn ihr PrintStatementDriven Development verwendet solltet ihr die print Statements rechtzeitig entfernen
        \end{itemize}
    \end{block}
\end{frame}

\begin{frame}{Tipps \& Tricks}
    \begin{block}{Hilfe}
        \begin{itemize}
            \item Unterstützt euch gegenseitig
            \item Teilen von Tests
            \item \href{https://codetester.ialistannen.de/}{Codetester}
            \item Fragen im Forum, etc.
        \end{itemize}
    \end{block}
\end{frame}

\begin{frame}{Und natürlich}
    \begin{block}{RTFM}
        \begin{center}
            \includegraphics[height=0.65\textheight]{rtfm.jpg}
        \end{center}
    \end{block}
\end{frame}
