\section{Variablen}

\begin{frame}{Variablen}
    \begin{block}{Allgemein}
        \begin{itemize}
            \item Eine \textbf{Variable} ist ein „Platzhalter“ für Werte eines Datentyps
            \item Name für einen Speicherplatz
        \end{itemize}
    \end{block}

    \pause

    \begin{block}{Deklaration}
        \begin{itemize}
            \item Notwendig vor der Verwendung
            \item Schema: \textit{Typ Name;}
            \item Beispiel: \textit{int student;}
        \end{itemize}
    \end{block}

    \pause

    \begin{block}{Zuweisung}
        \begin{itemize}
            \item Setzen, bzw. ändern, eines Wertes
            \item Schema: \textit{Name = Wert;}
            \item Beispiel: \textit{student = 1337;}
        \end{itemize}
    \end{block}
\end{frame}

\begin{frame}{Variablen}
    \begin{block}{Initialisierung}
        \begin{itemize}
            \item Kombination aus Deklaration und Zuweisung
            \item Schema: \textit{Typ Name = Wert;}
            \item Beispiel: \textit{int student = 1337;}
        \end{itemize}
    \end{block}

    \pause

    \begin{block}{Verwendung}
        \begin{itemize}
            \item Global: Attribute in Klassen (Zugriff mittels \textit{Objekt.Attribut})
            \item Lokal: innerhalb von Methoden
        \end{itemize}
    \end{block}
\end{frame}

\begin{frame}{Variablen}
    \begin{block}{\textit{null}}
        \begin{itemize}
            \item Spezieller Wert für „kein Objekt“
            \item Kennzeichnet nicht initialisierte (komplexe) Referenzvariablen
            \item \textit{NullPointerException} bei fälschlichem Zugriff
        \end{itemize}
    \end{block}
\end{frame}

\subsection{Konstanten}

\begin{frame}{Konstanten}
    \begin{block}{Allgemein}
        \begin{itemize}
            \item Analog zu Variablen
            \item \dots jedoch später nicht veränderbar.

                  \pause

            \item daher auch das Schlüsselwort \textit{final}
        \end{itemize}
    \end{block}

    \pause

    \begin{block}{Lokale Konstanten}
        \begin{itemize}
            \item Schema: \textit{final Typ Name = Wert;}
            \item Beispiel: \textit{final boolean marvin = false;}
        \end{itemize}
    \end{block}
\end{frame}

\begin{frame}{Konstanten}
    \begin{block}{Globale Konstanten}
        \begin{itemize}
            \item sogenannte Klassenkonstanten
            \item Schema: \textit{static final Typ NAME = Wert;}
            \item Beispiel: \textit{static final float LOVELY\_NUMBER = 3.14159265f;}

                  \pause

            \item Unterschied erkennbar zu normalen Variablen?

                  \pause

            \item \textbf{Großbuchstaben} und Trennung mit Unterstrich
        \end{itemize}
    \end{block}
\end{frame}
