\section{Sichtbarkeit, Getter / Setter}

\subsection{Sichtbarkeit}

\begin{frame}{Sichtbarkeit}
    \begin{block}{Sichtbarkeit}
        Es ist wichtig, Attribute und Methoden vor externen Zugriffen zu schützen, bzw. diesen Zugriff bewusst zuzulassen. Dafür gibt es die folgenden Schlüsselworte:

        \pause

        \begin{center}
            \begin{tabular}{lcccc}
                \textbf{Modifier}      & \textbf{Klasse} & \textbf{Paket} & \textbf{Unterklasse} & \textbf{Welt} \\
                \midrule[1pt]
                \textbf{public}        & Y               & Y              & Y                    & Y             \\
                \cmidrule[0.5pt]{2-5}
                \textbf{protected}     & Y               & Y              & Y                    & N             \\
                \cmidrule[0.5pt]{2-5}
                \textbf{<no modifier>} & Y               & Y              & N                    & N             \\
                \cmidrule[0.5pt]{2-5}
                \textbf{private}       & Y               & N              & N                    & N             \\
            \end{tabular}
        \end{center}
    \end{block}
\end{frame}

\begin{frame}{Sichtbarkeit - Merkregeln}
    \begin{block}{Sichtbarkeit - Merkregeln für unser Modul}
        \begin{itemize}
            \item Alles hat einen Modifier (nein, <no modifier> ist kein Modifier)
            \item \textit{private} für Attribute, Hilfsmethoden, private Konstruktoren
            \item \textit{public} definiert Schnittstellen zu anderen Klassen
            \item \textit{protected} falls notwendig \\ \ \\

                  \pause

            \item \textbf{Beginne immer mit \textit{private} und lockere die Sichtbarkeit}.
        \end{itemize}
    \end{block}
\end{frame}

\begin{frame}{Sichtbarkeit - Beispiel}
    \begin{exampleblock}{Beispiel}
        \lstinputlisting[basicstyle=\small]{../modules/visibility/src/visibility.java}
    \end{exampleblock}

    \begin{block}{Und nun?}
        Wie können wir aber auf solche \textit{private} Attribute / Methoden zugreifen?
    \end{block}
\end{frame}

\subsection{Getter / Setter}

\begin{frame}{Getter / Setter}
    \begin{block}{Getter / Setter}
        \textbf{Die Lösung}: sogenannte Getter und Setter! \\
        \textbf{Getter} helfen beim kontrollierten Auslesen eines private Attributes \\
        \textbf{Setter} helfen beim kontrollierten Setzen eines private Attributes
    \end{block}

    \begin{alertblock}{Grundsätzlich \dots}
        \dots sollten Attribute einer Klasse nur über \textit{Getter} und \textit{Setter} zugreifbar sein. \\
        Setter ermöglichen auch, Plausibilitätsabfragen zu implementieren und mögliche Fehlerzustände zu verhindern.
    \end{alertblock}
\end{frame}

\begin{frame}{Getter / Setter - Beispiel}
    \begin{exampleblock}{Beispiel}
        \lstinputlisting[basicstyle=\footnotesize]{../modules/visibility/src/getter-setter.java}
    \end{exampleblock}
\end{frame}
